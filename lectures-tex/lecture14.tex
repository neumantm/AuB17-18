\documentclass{scrartcl}% siehe <http://www.komascript.de>
% packages ---------------------------------------------------------------------------------------------------- packages
\usepackage[margin=3cm]{geometry}
\usepackage[utf8]{inputenc}
\usepackage[T1]{fontenc}
\usepackage[ngerman]{babel}
\usepackage{lmodern}
\usepackage{amsmath}
\usepackage{amssymb}
\usepackage{amsfonts}
\usepackage{cancel}
\usepackage{listings}
\usepackage{multirow}
\usepackage{hhline}
\usepackage{tikz}
\usepackage{pgfplots}
\usepackage[all,dvips,arc,curve,color,frame]{xy}
\usepackage[yyyymmdd,hhmm]{datetime}
\usepackage{float}
\usepackage{hyperref}
\usepackage[table]{xcolor,colortbl}
\usepackage{graphicx}
\usepackage{subcaption}
\usepackage{tabularx}
\usepackage{footnote}
\usepackage{multicol}
\usepackage{stackengine}
\usepackage[norndcorners,customcolors,shade]{hf-tikz}
\usepackage{siunitx}
\usepackage{algorithm}
\usepackage[]{algorithmic}
\usepackage{centernot}
\usepackage{amsthm}
% \usepackage[noend]{algorithmic}

% pgf plotsets -------------------------------------------------------------------------------------------- pgf plotsets
\pgfplotsset{ticks=none}
\pgfplotsset{width=10cm, compat=1.5.1}

% tikz libraries ---------------------------------------------------------------------------------------- tikz libraries
\usetikzlibrary{shapes}
\usetikzlibrary{positioning}
\usetikzlibrary{shapes.misc}
\usetikzlibrary{shapes.arrows,calc,quotes,babel}
\usetikzlibrary{positioning,arrows,chains,scopes,fit}
\usetikzlibrary{matrix,backgrounds}
\usetikzlibrary{trees}
\usetikzlibrary{calc}
\usetikzlibrary{tikzmark}
\usetikzlibrary{intersections}

\sisetup{
locale = DE ,
per-mode = symbol
}

\algsetup{indent=2em}
\newlength\myindent
\setlength\myindent{2em}
\newcommand\bindent{%
    \begingroup
    \setlength{\itemindent}{\myindent}
    \addtolength{\algorithmicindent}{\myindent}
}
\newcommand\eindent{\endgroup}

% commands ---------------------------------------------------------------------------------------------------- commands
\setlength{\parindent}{0pt}
\newcommand{\includepic}[2]{\includegraphics[width=#2\textwidth,height=#2\textheight,keepaspectratio]{#1}}
\newcommand\circlearound[1]{%
\tikz[baseline]\node[draw,shape=circle,anchor=base] {#1} ;}
\newcommand{\emptyrow}[0]{\multicolumn{1}{c}{} & \multicolumn{1}{c}{} & \multicolumn{1}{c}{} & \multicolumn{1}{c}{} &
\multicolumn{1}{c}{} & \multicolumn{1}{c}{} & \multicolumn{1}{c}{} & \multicolumn{1}{c}{} &
\multicolumn{1}{c}{} & \multicolumn{1}{c|}{} & & \\}
\newcommand\underbracetext[1]{\raisebox{1ex}{\ensuremath{\underbrace{\hphantom{\text{#1}}}_{\text{\normalsize #1}}}}}
\newcommand{\headerline}[3]{
\subject{#2}
\title{#1}
\subtitle{#3}
\date{Letztes Update: \today \ - \currenttime \ Uhr}
\maketitle
}
\newcommand{\header}[3]{\section*{#1: #3 - #2}\subsection*{Letztes Update: \today \ - \currenttime \ Uhr}\vspace*{1cm}}
\newcommand{\newproof}[2]{
\vspace*{0.3cm} \textbf{\textsf{Satz:}} #1

\vspace*{0.3cm} \textbf{\textsf{Beweis:}} #2 \vspace*{0.3cm}
}
%\newcommand{\uparrow}[0]{\rlap{\scalebox{1.6}{$\uparrow$}}}
\newlength\dlf
\newcommand{\proofend}[0]{\hfill $\square$}
\newcommand\drawbi[7][4pt]{%
    \path(#2)--(#3)coordinate[at start](h1)coordinate[at end](h2);
    \draw[#4]($(h1)!#1!90:(h2)$)-- node [auto=left] {#5} ($(h2)!#1!-90:(h1)$);
    \draw[#6]($(h1)!#1!-90:(h2)$)-- node [auto=right] {#7} ($(h2)!#1!90:(h1)$);
    }

% colors -------------------------------------------------------------------------------------------------------- colors
\definecolor{amethyst}{rgb}{0.6, 0.4, 0.8}
\definecolor{anti-flashwhite}{rgb}{0.95, 0.95, 0.96}
\definecolor{applegreen}{rgb}{0.55, 0.71, 0.0}
\definecolor{babyblue}{rgb}{0.54, 0.81, 0.94}
\definecolor{bistre}{rgb}{0.24, 0.17, 0.12}
\definecolor{buff}{rgb}{0.94, 0.86, 0.51}
\definecolor{dodgerblue}{rgb}{0.12, 0.56, 1.0}
\definecolor{icterine}{rgb}{0.99, 0.97, 0.37}
\definecolor{tan}{rgb}{0.82, 0.71, 0.55}
\definecolor{upforestgreen}{rgb}{0.0, 0.27, 0.13}

% listing colors
\definecolor{listingBlue}{rgb}{0.13,0.13,1}
\definecolor{listingGreen}{rgb}{0,0.5,0}
\definecolor{listingRed}{rgb}{0.9,0,0}
\definecolor{listingGrey}{rgb}{0.46,0.45,0.48}

% listings ---------------------------------------------------------------------------------------------------- listings
\lstset{basicstyle=\ttfamily}
\lstset{literate=%
{Ö}{{\"O}}1
{Ä}{{\"A}}1
{Ü}{{\"U}}1
{ß}{{\ss}}1
{ü}{{\"u}}1
{ä}{{\"a}}1
{ö}{{\"o}}1
}
\lstset{language=Java,
showspaces=false,
showtabs=false,
breaklines=true,
showstringspaces=false,
breakatwhitespace=true,
commentstyle=\color{listingGreen},
keywordstyle=\color{listingBlue},
stringstyle=\color{listingRed},
basicstyle=\ttfamily,
moredelim=[il][\textcolor{listingGrey}]{$$},
moredelim=[is][\textcolor{listingGrey}]{\%\%}{\%\%}
}
\lstdefinelanguage{Kotlin}{
  keywords={package, as, typealias, this, super, val, var, fun, for, null, true, false, is, in, throw, return, break, continue, object, if, try, else, while, do, when, yield, typeof, yield, typeof, class, interface, enum, object, override, public, private, get, set, import, abstract, procedure},
  keywordstyle=\color{listingBlue}\bfseries,
  ndkeywords={@Deprecated, Iterable, Int, Integer, Float, Double, String, Runnable, dynamic},
  ndkeywordstyle=\color{amethyst}\bfseries,
  emph={println, return@, forEach,},
  emphstyle={\color{orange}},
  identifierstyle=\color{black},
  sensitive=true,
  commentstyle=\color{listingGreen}\ttfamily,
  comment=[l]{//},
  morecomment=[s]{/*}{*/},
  stringstyle=\color{listingRed}\ttfamily,
  morestring=[b]",
  morestring=[s]{"""*}{*"""},
}

% misc ------------------------------------------------------------------------------------------------------------ misc
\graphicspath{ {../lectures-img/} }
\setlength{\parindent}{0pt}

\newcolumntype{C}[1]{>{\centering\arraybackslash}p{#1}} % zentriert mit Breitenangabe
\newcolumntype{L}[1]{>{\raggedright\arraybackslash}p{#1}} % rechtsbündig mit Breitenangabe
\newcolumntype{R}[1]{>{\raggedleft\arraybackslash}p{#1}} % rechtsbündig mit Breitenangabe

% text ------------------------------------------------------------------------------------------------------------ text
\newcommand{\fett}[1]{\textbf{\textsf{#1}}}
\newcommand{\kursiv}[1]{\textit{#1}}

% proofs -------------------------------------------------------------------------------------------------------- proofs
\newtheoremstyle{dotless}{}{}{\itshape}{}{\bfseries\sffamily}{}{ }{}
\theoremstyle{dotless}
\newtheorem*{satz}{Satz:}
\renewenvironment{proof}{{\bfseries \sffamily Beweis:}}{\hfill $\square$}


\usepackage{diagbox}

\definecolor{cadmiumred}{rgb}{0.89, 0.0, 0.13}
\definecolor{cadmiumorange}{rgb}{0.93, 0.53, 0.18}
\definecolor{cadmiumyellow}{rgb}{1.0, 0.96, 0.0}
\definecolor{cadmiumgreen}{rgb}{0.0, 0.42, 0.24}
\definecolor{blizzardblue}{rgb}{0.67, 0.9, 0.93}
\definecolor{persianblue}{rgb}{0.11, 0.22, 0.73}
\definecolor{darkviolet}{rgb}{0.58, 0.0, 0.83}

\begin{document}
    \headerline{Algorithmen und Berechenbarkeit}{Vorlesungsmitschrift}{Vorlesung 14}

    \section*{Unentscheidbarkeit der Diagonalsprache $\mathcal{D}$}
    \label{sec:unentscheidbarkeitDerDiagonalsprache}
    Es sei die Diagonalsprache $\mathcal{D}$ folgendermaßen definiert:

    \begin{equation*}
        \mathcal{D} = \{ w \in \{0,1\}^* \ | \ w =w_i \land M_i \text{ akzeptiert } w \text{ nicht} \}
    \end{equation*}
    $\mathcal{D}$ ist also so definiert,
    dass ein $i$-tes Wort genau dann enthalten ist, wenn die $i$-te TM es nicht akzeptiert.

    \begin{table}[H]
        \centering
        \small
        \begin{tabular}{C{1.5em}|C{1.5em}C{1.5em}C{1.5em}C{1.5em}C{1.5em}C{1.5em}C{1.5em}C{1.5em}C{1.5em}}
            \tiny Wort & \tiny $\epsilon$ & \tiny\texttt{0} & \tiny\texttt{1} & \tiny\texttt{00} & \tiny\texttt{01} & \tiny\texttt{10} & \tiny\texttt{11} & \tiny$\cdots$ \\
            & $w_0$  & $w_1$ & $w_2$ & $w_3$ & $w_4$ & $w_5$ & $w_6$ & \tiny$\cdots$ \\ \hline
            $\mathcal{M}_0$  & \cellcolor{cadmiumgreen!20}$0$  & & & & & & & \\
            $\mathcal{M}_1$  & & \cellcolor{cadmiumgreen!20}$0$ & & & & & & \\
            $\mathcal{M}_2$  & & & $1$ & & & & & \\
            $\mathcal{M}_3$  & & & & $1$ & & & & \\
            $\mathcal{M}_4$  & & & & & \cellcolor{cadmiumgreen!20}$0$ & & & \\
        \end{tabular}
    \end{table}

    Beim betrachten der Tabelle fällt auf, dass $\mathcal{D}$ auch so definiert werden kann:
    \begin{equation*}
        \mathcal{D} = \{ w_i \ | \ A_{i,i} = 0 \}
    \end{equation*}

    \textbf{\textsf{Theorem:}} Die Diagonalsprache $\mathcal{D}$ ist nicht entscheidbar.

    \vspace*{0.3cm}
    \textbf{\textsf{Beweis (durch Widerspruch):}} Sie $\mathcal{M}_j$ eine TM, die $\mathcal{D}$ entscheidet.
    Man lässt $\mathcal{M}_j$ auf der Eingabe $w_j$ laufen und erhält zwei Fälle.

    \begin{enumerate}
        \item $\mathcal{M}_j$ akzeptiert $w_j$. Dann gilt: $  w_j \in \mathcal{D}$ \newline
        Aber dadurch gilt auch $A_{j,j} = 1$. Dann gilt: $w_j \notin \mathcal{D}$
        \item $\mathcal{M}_j$ verwirft $w_j$. Dann gilt: $w_j \notin \mathcal{D}$ \newline
        Aber dadurch gilt auch $A_{j,j} = 0$. Dann gilt: $w_j \in \mathcal{D}$
    \end{enumerate}

    Eine solche TM kann es nicht geben.\proofend

    \vspace*{0.3cm}
    \textbf{\textsf{Satz:}} Sei $\mathcal{L}$ eine unentscheidbare Sprache,
    dann ist das Komplement $\overline{\mathcal{L}} = \Sigma^* - \mathcal{L}$ auch unentscheidbar.

    \vspace*{0.3cm}
    \textbf{\textsf{Beweis:}} Man nutzt eine Maschine, die $\overline{\mathcal{L}}$ entscheidet,
    um $\mathcal{L}$ zu entscheiden, sodass man \texttt{JA} und \texttt{NEIN} vertauscht.
    Falls $\overline{\mathcal{L}}$ entscheidbar, so ist auch $\mathcal{L}$ entscheidbar.\proofend

    \section*{Unentscheidbarkeit des Halteproblems}
    Vom Halteproblem spricht man, wenn man ein gegebenes Programm samt Eingabe für dieses Programm laufen lässt
    und entscheiden muss, ob das Programm auf dieser Eingabe terminiert.

    Formal also
    \begin{equation*}
        \mathcal{H} = \{ <M>w \ | \ M \text{ hält auf } w \}
    \end{equation*}

    Um die Unentscheidbarkeit zu zeigen, sollte sich vorher klargemacht werden,
    was \textit{semientscheidbar} bzw. \textit{Akzeptieren} bedeutet:
    \textit{Eine TM akzeptiert ein Wort, falls das Wort immer richtig von der TM erkannt wird.
    Wörter, die nicht in der Sprache sind, können von der TM jedoch nicht als solche erkannt werden.}

    \vspace*{0.3cm}
    Auch die universelle TM $\mathcal{U}$ kann $\mathcal{H}$ nicht entscheiden.
    Zwar könnte für alle $<M>w \in \mathcal{H}$ die TM $\mathcal{U}$ irgendwann \texttt{JA} sagen,
    aber nie mit Sicherheit \texttt{NEIN}.

    \vspace*{0.5cm}
    \textbf{\textsf{Theorem:}} Das Halteproblem bzw. die Sprache $\mathcal{H}$ ist unentscheidbar.

    \vspace*{0.3cm}
    \textbf{\textsf{Beweis:}} Angenommen, $\mathcal{H}$ ist durch die TM $\mathcal{M}_\mathcal{H}$ entscheidbar.
    Dann muss es möglich sein, aus $\mathcal{M}_\mathcal{H}$ eine TM  $\mathcal{M}_\mathcal{D}$ zu konstruieren, die
    $\overline{D}$ (und damit auch $\mathcal{D}$) entscheidet.

    So eine TM $\mathcal{M}_\mathcal{D}$ kann es aber nicht geben, also kann es auch die TM $\mathcal{M}_\mathcal{H}$ nicht geben.

    \vspace*{0.3cm}
    $\mathcal{M}_{\overline{\mathcal{D}}}$ müsste folgendermaßen aussehen:
    \begin{enumerate}
        \item Man berechnet $i$ auf der Eingabe $w$ sodass gilt $w = w_i$.
        \item Man berechnet $<M_{i}>$ der $i$-ten TM.
        \item Man startet $\mathcal{M}_\mathcal{H}$ als Unterprogramm auf der Eingabe $<M_i>w$.
        \item Falls $\mathcal{M}_\mathcal{H}$ akzeptiert, wird $w$ auf $\mathcal{M}_i$ simuliert und die Ausgabe übernommen.
        \item Falls $\mathcal{M}_\mathcal{H}$ verwirft, so wird die Eingabe verworfen.
    \end{enumerate}

    Korrektheit von $\mathcal{M}_{\overline{\mathcal{D}}}$:
    \begin{enumerate}
        \item $w \in \overline{\mathcal{D}} \Rightarrow \mathcal{M}_\mathcal{H}$ akzeptiert $<M_{i}>w$ im dritten Schritt
        (da $M_i$ das $w=w_i$ im vierten Schritt akzeptiert und damit auch stoppt)
        $\Rightarrow \mathcal{M}_{\overline{\mathcal{D}}}$ akzeptiert $w$.
        \item $w \in \overline{\mathcal{D}} \Rightarrow$
        \begin{itemize}
            \item $\mathcal{M}_\mathcal{H}$ verwirft $<M_{i}>w$ im dritten Schritt, da $M_i$ auf $w$ nicht hält.
            \newline $\Rightarrow \mathcal{M}_{\overline{\mathcal{D}}}$ verwirft $w$.
            \item $\mathcal{M}_\mathcal{H}$ akzeptiert $<M_{i}>w$ im dritten Schritt, aber $M_i$ verwirft $w$.
            \newline $\Rightarrow \mathcal{M}_{\overline{\mathcal{D}}}$ verwirft $w$.
        \end{itemize}
    \end{enumerate}\proofend

    \newpage
    \section*{Das Spezielle Halteproblem}
    Das Spezielle Halteproblem ist definiert als
    \begin{equation*}
        \mathcal{H_\epsilon} = \{ <M> \ | \ M \text{ hält auf der Eingabe } \epsilon \}
    \end{equation*}

    \vspace*{0.5cm}
    \textbf{\textsf{Satz:}} Das Spezielle Halteproblem $\mathcal{H}_\epsilon$ ist unentscheidbar.

    \vspace*{0.3cm}
    \textbf{\textsf{Beweis:}} Die Idee ist wieder wie vorhin: Man nimmt an, es gibt eine TM $\mathcal{M}_\epsilon$,
    die $\mathcal{H}_\epsilon$ entscheidet. Dann kann $\mathcal{M}_\epsilon$ als Unterprogramm einer TM genutzt werden,
    die $\mathcal{H}_\epsilon$ entscheidet. So eine $\mathcal{M}_\epsilon$ kann es aber nicht geben. Als Folge würde dann gelten:
    \begin{equation*}
        \begin{align}
            \Rightarrow \ & \mathcal{M}_\epsilon & \text{ kann nicht existieren.}\\
            \Rightarrow \ & \mathcal{H}_\epsilon & \text{ ist unentscheidbar.}\\
        \end{align}
    \end{equation*}

    Die TM $\mathcal{M}_{\mathcal{H}}$ kann beschrieben werden:
    \begin{enumerate}
        \item Falls die Eingabe mit einer Kodierung einer TM beginnt, so wird die Eingabe verworfen.
        \item Falls die Eingabe $<M>w$ ist, so soll eine TM $\mathcal{M}_w$ konstruiert werden, die Folgendes macht:
        \begin{itemize}
            \item Falls $\mathcal{M}_w$ auf keiner Eingabe startet, schreibt man $w$ auf das Band und simuliert $\mathcal{M}$ auf $w$.
            \item Auf anderen Eingaben kann sich $\mathcal{M}_w$ beliebig verhalten.
        \end{itemize}
        \item Man lässt $\mathcal{M}_\epsilon$ entscheiden, ob $\mathcal{M}_w$ auf dem leeren Wort terminiert.
        Falls terminiert, entscheidet man mit \texttt{JA}, sonst mit \texttt{NEIN}.
    \end{enumerate}

    Korrektheit der Konstruktion:
    \begin{itemize}
        \item $<M>w \in \mathcal{H}$
        \begin{itemize}
            \item [$\Rightarrow$] $\mathcal{M}$ hält auf $w$
            \item [$\Rightarrow$] $\mathcal{M}_w$ hält auf $\epsilon$
            \item [$\Rightarrow$] $\mathcal{M}_\epsilon$ akzeptiert $<M_w>$
            \item [$\Rightarrow$] $\mathcal{M}_\mathcal{H}$ akzeptiert $<M>w$
        \end{itemize}
        \item $<M>w \notin \mathcal{H}$
        \begin{itemize}
            \item [$\Rightarrow$] $\mathcal{M}$ hält nicht auf $w$
            \item [$\Rightarrow$] $\mathcal{M}_w$ hält nicht auf $\epsilon$
            \item [$\Rightarrow$] $\mathcal{M}_\epsilon$ verwirft $<M_w>$
            \item [$\Rightarrow$] $\mathcal{M}_\mathcal{H}$ verwirft $<M>w$
        \end{itemize}
    \end{itemize}

%    \textit{Anmerkung}:
    %    \begin{itemize}
    %        \item Ist eine Sprache und sein Komplement jeweils semientscheidbar, so ist die Sprache entscheidbar.
    %        \item Ist eine Sprache semientscheidbar, sein Komplement aber nicht mal semientscheidbar, so ist die Sprache unentscheidbar.
    %    \end{itemize}

\end{document}