\documentclass{scrartcl}% siehe <http://www.komascript.de>
% packages ---------------------------------------------------------------------------------------------------- packages
\usepackage[margin=3cm]{geometry}
\usepackage[utf8]{inputenc}
\usepackage[T1]{fontenc}
\usepackage[ngerman]{babel}
\usepackage{lmodern}
\usepackage{amsmath}
\usepackage{amssymb}
\usepackage{amsfonts}
\usepackage{cancel}
\usepackage{listings}
\usepackage{multirow}
\usepackage{hhline}
\usepackage{tikz}
\usepackage{pgfplots}
\usepackage[all,dvips,arc,curve,color,frame]{xy}
\usepackage[yyyymmdd,hhmm]{datetime}
\usepackage{float}
\usepackage{hyperref}
\usepackage[table]{xcolor,colortbl}
\usepackage{graphicx}
\usepackage{subcaption}
\usepackage{tabularx}
\usepackage{footnote}
\usepackage{multicol}
\usepackage{stackengine}
\usepackage[norndcorners,customcolors,shade]{hf-tikz}
\usepackage{siunitx}
\usepackage{algorithm}
\usepackage[]{algorithmic}
\usepackage{centernot}
\usepackage{amsthm}
% \usepackage[noend]{algorithmic}

% pgf plotsets -------------------------------------------------------------------------------------------- pgf plotsets
\pgfplotsset{ticks=none}
\pgfplotsset{width=10cm, compat=1.5.1}

% tikz libraries ---------------------------------------------------------------------------------------- tikz libraries
\usetikzlibrary{shapes}
\usetikzlibrary{positioning}
\usetikzlibrary{shapes.misc}
\usetikzlibrary{shapes.arrows,calc,quotes,babel}
\usetikzlibrary{positioning,arrows,chains,scopes,fit}
\usetikzlibrary{matrix,backgrounds}
\usetikzlibrary{trees}
\usetikzlibrary{calc}
\usetikzlibrary{tikzmark}
\usetikzlibrary{intersections}

\sisetup{
locale = DE ,
per-mode = symbol
}

\algsetup{indent=2em}
\newlength\myindent
\setlength\myindent{2em}
\newcommand\bindent{%
    \begingroup
    \setlength{\itemindent}{\myindent}
    \addtolength{\algorithmicindent}{\myindent}
}
\newcommand\eindent{\endgroup}

% commands ---------------------------------------------------------------------------------------------------- commands
\setlength{\parindent}{0pt}
\newcommand{\includepic}[2]{\includegraphics[width=#2\textwidth,height=#2\textheight,keepaspectratio]{#1}}
\newcommand\circlearound[1]{%
\tikz[baseline]\node[draw,shape=circle,anchor=base] {#1} ;}
\newcommand{\emptyrow}[0]{\multicolumn{1}{c}{} & \multicolumn{1}{c}{} & \multicolumn{1}{c}{} & \multicolumn{1}{c}{} &
\multicolumn{1}{c}{} & \multicolumn{1}{c}{} & \multicolumn{1}{c}{} & \multicolumn{1}{c}{} &
\multicolumn{1}{c}{} & \multicolumn{1}{c|}{} & & \\}
\newcommand\underbracetext[1]{\raisebox{1ex}{\ensuremath{\underbrace{\hphantom{\text{#1}}}_{\text{\normalsize #1}}}}}
\newcommand{\headerline}[3]{
\subject{#2}
\title{#1}
\subtitle{#3}
\date{Letztes Update: \today \ - \currenttime \ Uhr}
\maketitle
}
\newcommand{\header}[3]{\section*{#1: #3 - #2}\subsection*{Letztes Update: \today \ - \currenttime \ Uhr}\vspace*{1cm}}
\newcommand{\newproof}[2]{
\vspace*{0.3cm} \textbf{\textsf{Satz:}} #1

\vspace*{0.3cm} \textbf{\textsf{Beweis:}} #2 \vspace*{0.3cm}
}
%\newcommand{\uparrow}[0]{\rlap{\scalebox{1.6}{$\uparrow$}}}
\newlength\dlf
\newcommand{\proofend}[0]{\hfill $\square$}
\newcommand\drawbi[7][4pt]{%
    \path(#2)--(#3)coordinate[at start](h1)coordinate[at end](h2);
    \draw[#4]($(h1)!#1!90:(h2)$)-- node [auto=left] {#5} ($(h2)!#1!-90:(h1)$);
    \draw[#6]($(h1)!#1!-90:(h2)$)-- node [auto=right] {#7} ($(h2)!#1!90:(h1)$);
    }

% colors -------------------------------------------------------------------------------------------------------- colors
\definecolor{amethyst}{rgb}{0.6, 0.4, 0.8}
\definecolor{anti-flashwhite}{rgb}{0.95, 0.95, 0.96}
\definecolor{applegreen}{rgb}{0.55, 0.71, 0.0}
\definecolor{babyblue}{rgb}{0.54, 0.81, 0.94}
\definecolor{bistre}{rgb}{0.24, 0.17, 0.12}
\definecolor{buff}{rgb}{0.94, 0.86, 0.51}
\definecolor{dodgerblue}{rgb}{0.12, 0.56, 1.0}
\definecolor{icterine}{rgb}{0.99, 0.97, 0.37}
\definecolor{tan}{rgb}{0.82, 0.71, 0.55}
\definecolor{upforestgreen}{rgb}{0.0, 0.27, 0.13}

% listing colors
\definecolor{listingBlue}{rgb}{0.13,0.13,1}
\definecolor{listingGreen}{rgb}{0,0.5,0}
\definecolor{listingRed}{rgb}{0.9,0,0}
\definecolor{listingGrey}{rgb}{0.46,0.45,0.48}

% listings ---------------------------------------------------------------------------------------------------- listings
\lstset{basicstyle=\ttfamily}
\lstset{literate=%
{Ö}{{\"O}}1
{Ä}{{\"A}}1
{Ü}{{\"U}}1
{ß}{{\ss}}1
{ü}{{\"u}}1
{ä}{{\"a}}1
{ö}{{\"o}}1
}
\lstset{language=Java,
showspaces=false,
showtabs=false,
breaklines=true,
showstringspaces=false,
breakatwhitespace=true,
commentstyle=\color{listingGreen},
keywordstyle=\color{listingBlue},
stringstyle=\color{listingRed},
basicstyle=\ttfamily,
moredelim=[il][\textcolor{listingGrey}]{$$},
moredelim=[is][\textcolor{listingGrey}]{\%\%}{\%\%}
}
\lstdefinelanguage{Kotlin}{
  keywords={package, as, typealias, this, super, val, var, fun, for, null, true, false, is, in, throw, return, break, continue, object, if, try, else, while, do, when, yield, typeof, yield, typeof, class, interface, enum, object, override, public, private, get, set, import, abstract, procedure},
  keywordstyle=\color{listingBlue}\bfseries,
  ndkeywords={@Deprecated, Iterable, Int, Integer, Float, Double, String, Runnable, dynamic},
  ndkeywordstyle=\color{amethyst}\bfseries,
  emph={println, return@, forEach,},
  emphstyle={\color{orange}},
  identifierstyle=\color{black},
  sensitive=true,
  commentstyle=\color{listingGreen}\ttfamily,
  comment=[l]{//},
  morecomment=[s]{/*}{*/},
  stringstyle=\color{listingRed}\ttfamily,
  morestring=[b]",
  morestring=[s]{"""*}{*"""},
}

% misc ------------------------------------------------------------------------------------------------------------ misc
\graphicspath{ {../lectures-img/} }
\setlength{\parindent}{0pt}

\newcolumntype{C}[1]{>{\centering\arraybackslash}p{#1}} % zentriert mit Breitenangabe
\newcolumntype{L}[1]{>{\raggedright\arraybackslash}p{#1}} % rechtsbündig mit Breitenangabe
\newcolumntype{R}[1]{>{\raggedleft\arraybackslash}p{#1}} % rechtsbündig mit Breitenangabe

% text ------------------------------------------------------------------------------------------------------------ text
\newcommand{\fett}[1]{\textbf{\textsf{#1}}}
\newcommand{\kursiv}[1]{\textit{#1}}

% proofs -------------------------------------------------------------------------------------------------------- proofs
\newtheoremstyle{dotless}{}{}{\itshape}{}{\bfseries\sffamily}{}{ }{}
\theoremstyle{dotless}
\newtheorem*{satz}{Satz:}
\renewenvironment{proof}{{\bfseries \sffamily Beweis:}}{\hfill $\square$}


\tikzset{node black/.style={circle,fill=black!100,draw,minimum size=0.05cm,inner sep=0pt},}
\begin{document}
    \header{Algorithmen und Berechenbarkeit}{Mitschrift}{Vorlesung 01}
    \section*{Randomisierte Algorithmen}

    \kursiv{Randomisierte Algorithmen} sind Algorithmen,
    die Probleme unter Einbeziehung einer Zufallsquelle (z.B.\ Münzwurf, Zufallsgenerator) lösen.
    In manchen Fällen sind diese Algorithmen einfacher und effizienter
    als ihre deterministischen Pendants.\\

    Randomisierte Algorithmen werden nach \kursiv{dem Verbrauch von Zufall} bewertet
    (im Gegensatz zu "`normalen"' Algorithmen, die nach Platz- und Zeitbedarf bewertet werden).
    Die "`Generierung"' echten Zufalls ist jedoch \fett{teuer}.

    \section*{Closest-Pair}

    Gegeben sei eine Menge $P$ von Punkten im $\mathbb{R}^2$.
    Finde $p_1,p_2 \in P$ für die $\vert p_{1},p_2 \vert $ (\kursiv{euklidische Distanz}) minimal.

    \begin{figure}[htb]
        \centering
        \framebox        {
        \begin{tikzpicture}[thick, scale=0.7]
            \node[node black] (n1) at (1,0) {};
            \node[node black] (n2) at (1,1.2) {};
            \node[node black] (n3) at (0,3) {};
            \node[node black] (n4) at (1.5,3) {};
            \node[node black] (n5) at (4.1,4.8) {};
            \node[node black] (n6) at (4,4.3) {};
            \node[node black] (n7) at (7.1,2.4) {};
            \node[node black] (n8) at (6.9,1.3) {};
            \node[node black] (n9) at (7,3,3) {};
            \node[node black] (n10) at (6,2.3) {};
            \node[node black] (n11) at (5,0) {};
            \node[node black] (n11) at (3,2.3) {};
        \end{tikzpicture}
        }
    \end{figure}

    \subsection*{Ansatz 1: Naiv}
    Man betrachtet $P_1$ und berechnet die Distanzen zu $P_2, P_3, \dots, P_n$,
    danach betrachtet man $P_2$ und berechnet die Distanzen zu $P_3, P_4, \dots, P_n$.
    Dies wird für alle verbleibenden Punkte gemacht.
    Es ergibt sich hieraus eine Laufzeit von

    \begin{equation*}
        \mathcal{O}\left(\sum_{i=1}^{n-1}\right) = \mathcal{O}\left( \frac{(n-1) \cdot n}{2} \right) = \mathcal{O}(n^2)
    \end{equation*}

    und den tabellarisch aufgelisteten Laufzeiten.

    \begin{table}[H]
        \centering
        \begin{tabular}{llll}
            \fett{n} & \fett{Rechnung} & & \fett{Ergebnis}\\
            \hline \\ [-2ex]
            $=100       $ & $100^2 \cdot 10^{-9}$       & $=10^{-5}$   & $\SI{10}{\us}$ \\
            $=1000      $ & $1000^2 \cdot 10^{-9}$      & $= 10^{-3}$    & \SI{1}{\ms} \\
            $=10000     $ & $10000^2 \cdot 10^{-9}$     & $= 10$         & \SI{10}{\s} \\
            $=1000000   $ & $1000000^2 \cdot 10^{-9}$   & $= 10^3$       & \SI{10}{\min} \\
            \hline
        \end{tabular}
        \caption*{1-GHz-Prozessor (1.000.000.000 Instruktionen pro Sekunde) für $\mathcal{O}(n^2)$}
    \end{table}

    \subsection*{Ansatz 2: Randomisiert (Las-Vegas)}

    Ein randomisierter Ansatz des Closest-Pair-Problems hat eine \kursiv{erwartete} Laufzeit von $\mathcal{O}(n)$
    und funktioniert folgendermaßen:\\

    Sei $P$ die Punktemenge mit $P_1, P_2, P_3, \dots, P_n$
    und sei $\delta_i$ die aktuelle Closest-Pair-Distanz.

    Für das \fett{Einfügen} eines neuen Punktes $P_{i+1}$ gilt allgemein:
    Die neue Closest-Pair-Distanz erhält man,
    wenn man die Abstände vom neuen Punkt zu allen bisherigen Punkten berechnet und überprüft,
    ob es einen noch kürzeren Abstand zweier Punkte gibt als $\delta_i$.\\

    Optimiert werden kann das Einfügen durch die folgende Überlegung:
    Da $\delta_i$ bekannt ist, kann nun ein Gitter mit einer Maschenweite (Breite der Zeilen/Spalten) von $\delta_i$
    erzeugt werden.

    \begin{figure}[H]
        \centering
        \begin{table}[H]
            \centering
            \scalebox{1.1}{%
            \begin{tabular}{l|l|l|l|l|l|l}
                & & & & $\cdot$ & &  \\ \hline
                & \cellcolor{yellow!50}&\cellcolor{yellow!50} &\cellcolor{yellow!50}$\cdot$ & & $\cdot$ &  \\ \hline
                &\cellcolor{yellow!50} & \cellcolor{orange!75}$\cdot$ &\cellcolor{yellow!50} & & $\cdot$ &  \\ \hline
                & \cellcolor{yellow!50}$\cdot$ &\cellcolor{yellow!50} &\cellcolor{yellow!50} & & &  \\ \hline
                & $\cdot$ & & & & &  \\ \hline
                & & & $\cdot$ & & &
            \end{tabular}
            }
        \end{table}
    \end{figure}

    Beim \fett{Einfügen} eines neuen Punktes wird zuerst der neue Punkt $P_{i+1}$ im Gitter (orange) lokalisiert.
    Anschließend werden alle Punkte in den angrenzenden Zellen (gelb)
    mit dem neuen Punkt $P_{i+1}$ verglichen. Gilt

    \begin{itemize}
        \item $\delta_{i+1} = \delta_i $ : Ist die neue Closest-Pair-Distanz identisch mit der alten
        (\fett{guter Fall}), so muss nichts weiter getan werden
        und es ergibt sich eine Laufzeit von $\mathcal{O}(1)$.
        \item $\delta_{i+1} < \delta_i $ : Ist die neue Closest-Pair-Distanz
        kleiner als die alte (\fett{schlechter Fall}),
        so muss das Gitter mit der Maschenweite $\delta_{i+1}$ neu aufgebaut werden.
        Im schlechtesten Fall ergibt sich somit eine Laufzeit von $\mathcal{O}(n^2)$.
    \end{itemize}

    Da es auch Fälle geben kann, in denen jeder neu eingefügte Punkt gleichzeitig auch ein neues Closest-Pair bildet,
    das Gitter also bei jedem Einfügeschritt neu aufgebaut werden muss (Liste, sortiert nach größtem Abstand zueinander),
    werden die Punkte in \kursiv{zufälliger Reihenfolge} eingefügt.
    So bleibt eine Wahrscheinlichkeit von $\frac{2}{i+1}$, dass der nächste Punkt ein Closest-Pair-Punkt ist.
    Die erwarteten Kosten des Einfügens \fett{pro Punkt} sind

    \begin{equation*}
        \leq \quad \underbrace{\frac{2}{i+1} \cdot \mathcal{O}(i)}_{\text{schlechter Fall}} +
        \underbrace{\mathcal{O}(1)}_{\substack{\text{guter} \\ \text{Fall}}}
        \quad = \quad \mathcal{O}(1) + \mathcal{O}(1)
        \quad = \quad \mathcal{O}(1)
    \end{equation*}

    Der Erwartungswert für das Einfügen ergibt sich zu

    \begin{align*}
        E\left[ \sum^{n}_{i=1}(\text{Kosten für Einfügen von }P_i)\right]
        &= \sum^{n}_{i=1}E[\text{Kosten für Einfügen von }P_i] \\\nonumber
        &= \sum^{n}_{i=1} \mathcal{O}(1) \quad = \quad \mathcal{O}(n)
    \end{align*}

    \begin{satz}
        Die Wahrscheinlichkeit, beim Einfügen eines neuen Punktes $P_{i+1}$ das Gitter neu aufbauen zu müssen,
        ist $< \frac{2}{i+1}$.
    \end{satz}
    \begin{proof}
        Das Gitter muss genau dann neu aufgebaut werden, wenn der neue Punkt $P_{i+1}$
        einer der beiden Punkte ist, welche das Closest-Pair in der Menge der ersten $i+1$ Punkte bestimmen.
        Jeder der ersten $i+1$ Punkte ist mit gleicher Wahrscheinlichkeit der $P_{i+1}$.
        Falls das Closest-Pair eindeutig ist, gilt
        \begin{equation*}
            \text{Pr}(\text{Gitter muss neu aufgebaut werden}) = \frac{2}{i}
        \end{equation*}

        \fett{Wichtig:} Wenn der Algorithmus länger braucht, hat das nichts mit der Eingabe zu tun.
        Der randomisierte Closest-Pair-Algorithmus berechnet \fett{immer} ein korrektes Resultat.
    \end{proof}

    \subsection*{Ansatz 3: Deterministisch}
    \begin{satz}
        Deterministisch Closest-Pair ist $\in \Omega(n \cdot \log(n))$.
    \end{satz}

    \begin{table}[!ht]
        \centering
        \begin{tabular}{llll}
            \fett{n} & \fett{Rechnung} & \fett{Ergebnis}\\
            \hline \\ [-2ex]
            $=1000000   $ & $10^6 \cdot 6 \cdot  10^{-9}$   & 6 ms \\
            \hline
        \end{tabular}
        \caption*{1-GHz-Prozessor für $\mathcal{O}(n \cdot \log(n))$}
    \end{table}

    \begin{proof}
        Man weiß, dass Elementuniqueness (gegeben $n$ Zahlen $\rightarrow$ man prüft, ob eine Zahl doppelt vorkommt)
        $\Omega(n \cdot \log(n))$ braucht.

        Wenn nun Closest-Pair vergleichsbasiert besser als $o(n \cdot \log(n))$ gelöst werden könnte,
        so könnte man auch Elementuniqueness in dieser Zeit lösen.
        Das gilt jedoch nur in einem anderen Rechenmodell (Randomisierung und Abrundung).
    \end{proof}

    \section*{Anhang}
    \label{sec:anhang}

    \subsubsection*{Zufallsvariable und Erwartungswert}
    \label{sec:zufallsvariableUndErwartungswert}
    Sei $X$ die Zufallsvariable \kursiv{Augenzahl beim Wurf} mit einem normalen Würfel.
    Der Erwartungswert berechnet sich zu

    \begin{align*}
        E[X] & = \sum \text{Ereignis} \cdot \text{Pr}(\text{Ereignis}) \\
        & = \frac{1}{6} \cdot 1 + \frac{1}{6} \cdot 2 + \frac{1}{6} \cdot 3 +
        \frac{1}{6} \cdot 4 + \frac{1}{6} \cdot 5 + \frac{1}{6} \cdot 6 \\
        & = 3,5
    \end{align*}

\end{document}