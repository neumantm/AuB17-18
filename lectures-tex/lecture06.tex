\documentclass{scrartcl}% siehe <http://www.komascript.de>
% packages ---------------------------------------------------------------------------------------------------- packages
\usepackage[margin=3cm]{geometry}
\usepackage[utf8]{inputenc}
\usepackage[T1]{fontenc}
\usepackage[ngerman]{babel}
\usepackage{lmodern}
\usepackage{amsmath}
\usepackage{amssymb}
\usepackage{amsfonts}
\usepackage{cancel}
\usepackage{listings}
\usepackage{multirow}
\usepackage{hhline}
\usepackage{tikz}
\usepackage{pgfplots}
\usepackage[all,dvips,arc,curve,color,frame]{xy}
\usepackage[yyyymmdd,hhmm]{datetime}
\usepackage{float}
\usepackage{hyperref}
\usepackage[table]{xcolor,colortbl}
\usepackage{graphicx}
\usepackage{subcaption}
\usepackage{tabularx}
\usepackage{footnote}
\usepackage{multicol}
\usepackage{stackengine}
\usepackage[norndcorners,customcolors,shade]{hf-tikz}
\usepackage{siunitx}
\usepackage{algorithm}
\usepackage[]{algorithmic}
\usepackage{centernot}
\usepackage{amsthm}
% \usepackage[noend]{algorithmic}

% pgf plotsets -------------------------------------------------------------------------------------------- pgf plotsets
\pgfplotsset{ticks=none}
\pgfplotsset{width=10cm, compat=1.5.1}

% tikz libraries ---------------------------------------------------------------------------------------- tikz libraries
\usetikzlibrary{shapes}
\usetikzlibrary{positioning}
\usetikzlibrary{shapes.misc}
\usetikzlibrary{shapes.arrows,calc,quotes,babel}
\usetikzlibrary{positioning,arrows,chains,scopes,fit}
\usetikzlibrary{matrix,backgrounds}
\usetikzlibrary{trees}
\usetikzlibrary{calc}
\usetikzlibrary{tikzmark}
\usetikzlibrary{intersections}

\sisetup{
locale = DE ,
per-mode = symbol
}

\algsetup{indent=2em}
\newlength\myindent
\setlength\myindent{2em}
\newcommand\bindent{%
    \begingroup
    \setlength{\itemindent}{\myindent}
    \addtolength{\algorithmicindent}{\myindent}
}
\newcommand\eindent{\endgroup}

% commands ---------------------------------------------------------------------------------------------------- commands
\setlength{\parindent}{0pt}
\newcommand{\includepic}[2]{\includegraphics[width=#2\textwidth,height=#2\textheight,keepaspectratio]{#1}}
\newcommand\circlearound[1]{%
\tikz[baseline]\node[draw,shape=circle,anchor=base] {#1} ;}
\newcommand{\emptyrow}[0]{\multicolumn{1}{c}{} & \multicolumn{1}{c}{} & \multicolumn{1}{c}{} & \multicolumn{1}{c}{} &
\multicolumn{1}{c}{} & \multicolumn{1}{c}{} & \multicolumn{1}{c}{} & \multicolumn{1}{c}{} &
\multicolumn{1}{c}{} & \multicolumn{1}{c|}{} & & \\}
\newcommand\underbracetext[1]{\raisebox{1ex}{\ensuremath{\underbrace{\hphantom{\text{#1}}}_{\text{\normalsize #1}}}}}
\newcommand{\headerline}[3]{
\subject{#2}
\title{#1}
\subtitle{#3}
\date{Letztes Update: \today \ - \currenttime \ Uhr}
\maketitle
}
\newcommand{\header}[3]{\section*{#1: #3 - #2}\subsection*{Letztes Update: \today \ - \currenttime \ Uhr}\vspace*{1cm}}
\newcommand{\newproof}[2]{
\vspace*{0.3cm} \textbf{\textsf{Satz:}} #1

\vspace*{0.3cm} \textbf{\textsf{Beweis:}} #2 \vspace*{0.3cm}
}
%\newcommand{\uparrow}[0]{\rlap{\scalebox{1.6}{$\uparrow$}}}
\newlength\dlf
\newcommand{\proofend}[0]{\hfill $\square$}
\newcommand\drawbi[7][4pt]{%
    \path(#2)--(#3)coordinate[at start](h1)coordinate[at end](h2);
    \draw[#4]($(h1)!#1!90:(h2)$)-- node [auto=left] {#5} ($(h2)!#1!-90:(h1)$);
    \draw[#6]($(h1)!#1!-90:(h2)$)-- node [auto=right] {#7} ($(h2)!#1!90:(h1)$);
    }

% colors -------------------------------------------------------------------------------------------------------- colors
\definecolor{amethyst}{rgb}{0.6, 0.4, 0.8}
\definecolor{anti-flashwhite}{rgb}{0.95, 0.95, 0.96}
\definecolor{applegreen}{rgb}{0.55, 0.71, 0.0}
\definecolor{babyblue}{rgb}{0.54, 0.81, 0.94}
\definecolor{bistre}{rgb}{0.24, 0.17, 0.12}
\definecolor{buff}{rgb}{0.94, 0.86, 0.51}
\definecolor{dodgerblue}{rgb}{0.12, 0.56, 1.0}
\definecolor{icterine}{rgb}{0.99, 0.97, 0.37}
\definecolor{tan}{rgb}{0.82, 0.71, 0.55}
\definecolor{upforestgreen}{rgb}{0.0, 0.27, 0.13}

% listing colors
\definecolor{listingBlue}{rgb}{0.13,0.13,1}
\definecolor{listingGreen}{rgb}{0,0.5,0}
\definecolor{listingRed}{rgb}{0.9,0,0}
\definecolor{listingGrey}{rgb}{0.46,0.45,0.48}

% listings ---------------------------------------------------------------------------------------------------- listings
\lstset{basicstyle=\ttfamily}
\lstset{literate=%
{Ö}{{\"O}}1
{Ä}{{\"A}}1
{Ü}{{\"U}}1
{ß}{{\ss}}1
{ü}{{\"u}}1
{ä}{{\"a}}1
{ö}{{\"o}}1
}
\lstset{language=Java,
showspaces=false,
showtabs=false,
breaklines=true,
showstringspaces=false,
breakatwhitespace=true,
commentstyle=\color{listingGreen},
keywordstyle=\color{listingBlue},
stringstyle=\color{listingRed},
basicstyle=\ttfamily,
moredelim=[il][\textcolor{listingGrey}]{$$},
moredelim=[is][\textcolor{listingGrey}]{\%\%}{\%\%}
}
\lstdefinelanguage{Kotlin}{
  keywords={package, as, typealias, this, super, val, var, fun, for, null, true, false, is, in, throw, return, break, continue, object, if, try, else, while, do, when, yield, typeof, yield, typeof, class, interface, enum, object, override, public, private, get, set, import, abstract, procedure},
  keywordstyle=\color{listingBlue}\bfseries,
  ndkeywords={@Deprecated, Iterable, Int, Integer, Float, Double, String, Runnable, dynamic},
  ndkeywordstyle=\color{amethyst}\bfseries,
  emph={println, return@, forEach,},
  emphstyle={\color{orange}},
  identifierstyle=\color{black},
  sensitive=true,
  commentstyle=\color{listingGreen}\ttfamily,
  comment=[l]{//},
  morecomment=[s]{/*}{*/},
  stringstyle=\color{listingRed}\ttfamily,
  morestring=[b]",
  morestring=[s]{"""*}{*"""},
}

% misc ------------------------------------------------------------------------------------------------------------ misc
\graphicspath{ {../lectures-img/} }
\setlength{\parindent}{0pt}

\newcolumntype{C}[1]{>{\centering\arraybackslash}p{#1}} % zentriert mit Breitenangabe
\newcolumntype{L}[1]{>{\raggedright\arraybackslash}p{#1}} % rechtsbündig mit Breitenangabe
\newcolumntype{R}[1]{>{\raggedleft\arraybackslash}p{#1}} % rechtsbündig mit Breitenangabe

% text ------------------------------------------------------------------------------------------------------------ text
\newcommand{\fett}[1]{\textbf{\textsf{#1}}}
\newcommand{\kursiv}[1]{\textit{#1}}

% proofs -------------------------------------------------------------------------------------------------------- proofs
\newtheoremstyle{dotless}{}{}{\itshape}{}{\bfseries\sffamily}{}{ }{}
\theoremstyle{dotless}
\newtheorem*{satz}{Satz:}
\renewenvironment{proof}{{\bfseries \sffamily Beweis:}}{\hfill $\square$}


\pgfplotsset{ticks=none}
\pgfplotsset{width=10cm, compat=1.5.1}

\begin{document}
    \headerline{Algorithmen und Berechenbarkeit}{Vorlesungsmitschrift}{Vorlesung 06}

    \section*{Randomisierte Skiplisten}\label{sec:skiplisten}
    \textsf{\textbf{Randomisierte Skipliste}} beschreibt eine Datenstruktur, die Ähnlichkeiten mit verketteten Listen aufweist.
    Die Elemente der Skipliste sind ebenfalls mit dem nächsten Element verkettet aber zusätzlich auch mit zufällig bestimmten anderen Elementen der Liste.

    Die drei Hauptoperationen \textit{Einfügen, Löschen} und \textit{Finden} werden unterstützt, wobei sich Letzteres wie folgt verhält:
    Die Suche nach $x$ liefert $x$, falls die Skipliste $x$ enthält, ansonsten wird $x'$ zurückgegeben, wobei $x'$ das nächstkleinere enthaltene Element von $x$ darstellt.

    Alternative Datenstrukturen werden aus verschiedenen Gründen nicht verwendet:
    \begin{itemize}
        \item Binärer Suchbaum und sortierte verkettete Listen \newline $\Rightarrow$ Suche in $\mathcal{O}(n)$ zu langsam
        \item Balancierter Suchbaum \newline $\Rightarrow$ zu kompliziert
        \item Sortiertes Array \newline $\Rightarrow$ \textit{Einfügen} und \textit{Löschen} mit $\mathcal{O}(n)$ zu langsam
    \end{itemize}

    \subsection*{Motivation}\label{subsec:motivation}
    \begin{figure}[H]
        \centering
        \tikzset{ arrow/.style={-latex, shorten >=0ex, shorten <=0ex}}
        \begin{tikzpicture}[draw, minimum width=1cm, minimum height=0.6cm]
            \node[draw] (n1) at (0,0) {4};
            \node[draw] (n2) at (2,0) {8};
            \node[draw] (n3) at (4,0) {15};
            \node[draw] (n4) at (6,0) {16};
            \node[draw] (n5) at (8,0) {23};
            \node[draw] (n6) at (10,0) {42};
            \node[draw] (n7) at (12,0) {43};

            \draw [arrow] (n1) to (n2);
            %            \draw [arrow, bend angle=45, bend left] (n1) to (n4);
            \draw [arrow, bend left] (n1) to (n4);
            \draw [arrow, bend left] (n1) to (n6);
            \draw [arrow] (n2) to (n3);
            \draw [arrow, bend right] (n2) to (n5);
            \draw [arrow] (n3) to (n4);
            \draw [arrow, bend right] (n3) to (n7);
            \draw [arrow] (n4) to (n5);
            \draw [arrow] (n5) to (n6);
            \draw [arrow] (n6) to (n7);
            %    \draw[-latex] (0.25,-1) .. controls (0.25,-1.25) and (1,-1.25) .. (out.north);
            %    \draw[-latex] (in.south) .. controls (-1, 1.5) and (-0.25,1.5) .. (-0.25,1);
        \end{tikzpicture}
    \end{figure}


    Durch zusätzliche Pointer wird ein schnelleres Bewegen durch die verkettete Liste ermöglicht.
    \newline $\Rightarrow$ Wie viele zusätzliche Pointer?
    \newline $\Rightarrow$ Wie weit die Sprünge?
    \newpage
    Jedes Element hat einen \textit{Turm} von Pointern, wobei die Höhe des Turmes zum Beispiel nach folgender Methode bestimmt werden kann:
    \begin{equation*}
        \begin{flalign}
            %            T(n) & \overset{\text{IV}}{\leq} \gamma \cdot \frac{n}{5} + \gamma \cdot \frac{7}{10} \cdot n + c \cdot n\\\nonumber
            P(\text{Höhe}=h) &= & 2^{-(h+1)} \\\nonumber
            &\Rightarrow & P(h=0) = \frac{1}{2} \\\nonumber
            & & P(h=1) = \frac{1}{4} \\\nonumber
            & & P(h=2) = \frac{1}{8} \\\nonumber
            & ... &   \\\nonumber
        \end{flalign}
    \end{equation*}
    Jede Turmetage wird mit dem nächsten Element des nächsten Turmes verbunden, der mindestens genauso groß ist.


    \begin{algorithm}
        \begin{algorithmic}
            \STATE Search($x$)
            \bindent
            \STATE $v\gets -\infty\text{-Turm}$
            \STATE $h\gets v.\text{height}$
            \STATE
            \WHILE{ $h > 0$}
                \WHILE {$x \geq v.\text{forward}[h] \rightarrow \text{key}$}
                    \STATE $v\gets v.\text{forward}[h]$
                \ENDWHILE
                \STATE $h\gets h-1$
            \ENDWHILE
            \STATE
            \RETURN $v$
            \eindent

        \end{algorithmic}
    \end{algorithm}

    \subsection*{Beispiel: Finde $44$}\label{subsec:beispiel:Finde}
    \begin{figure}[H]
        \centering
        \tikzset{ arrow/.style={-latex, gray}}
        \tikzset{ arrowom/.style={stealth-, orange}}
        \tikzset{ arrowr/.style={-stealth, red}}
        \tikzset{ arrowg/.style={-stealth, green}}
        \tikzset{ arrowgd/.style={-stealth, shorten >=0.5ex, shorten <=0.5ex, green}}
        \begin{tikzpicture}[draw, minimum width=1cm, minimum height=0.6cm]

            \node[draw] (n1_0) at (0,0) {$-\infty$};
            \node[draw] (n1_1) at (0,0.6) {};
            \node[draw] (n1_2) at (0,1.2) {};
            \node[draw] (n1_3) at (0,1.8) {};
            \node[draw] (n1_4) at (0,2.4) {};
            \node[draw] (n1_5) at (0,3.0) {};
            \node[draw] (n1_6) at (0,3.6) {};
            \draw[arrowom] (n1_6) -- node[sloped, ]         {}  +(150:1);

            \node[draw] (n2_0) at (1.5,0) {7};
            \node[draw] (n2_1) at (1.5,0.6) {};
            \node[draw] (n2_2) at (1.5,1.2) {};
            \node[draw] (n2_3) at (1.5,1.8) {};

            \node[draw] (n3_0) at (3,0) {15};

            \node[draw] (n4_0) at (4.5,0) {23};
            \node[draw] (n4_1) at (4.5,0.6) {};

            \node[draw] (n5_0) at (6,0) {36};
            \node[draw] (n5_1) at (6,0.6) {};
            \node[draw] (n5_2) at (6,1.2) {};
            \node[draw] (n5_3) at (6,1.8) {};
            \node[draw] (n5_4) at (6,2.4) {};

            \node[draw] (n6_0) at (7.5,0) {44};
            \node[draw] (n6_1) at (7.5,0.6) {};
            \node[draw] (n6_2) at (7.5,1.2) {};
            \node[draw] (n6_3) at (7.5,1.8) {};
            \node[draw] (n6_4) at (7.5,2.4) {};

            \node[draw] (n7_0) at (9,0) {99};
            \node[draw] (n7_1) at (9,0.6) {};

            \node[draw] (n8_0) at (10.5,0) {100};
            \node[draw] (n8_1) at (10.5,0.6) {};
            \node[draw] (n8_2) at (10.5,1.2) {};
            \node[draw] (n8_3) at (10.5,1.8) {};

            \node[draw] (n9_0) at (12,0) {$\infty$};
            \node[draw] (n9_1) at (12,0.6) {};
            \node[draw] (n9_2) at (12,1.2) {};
            \node[draw] (n9_3) at (12,1.8) {};
            \node[draw] (n9_4) at (12,2.4) {};
            \node[draw] (n9_5) at (12,3.0) {};
            \node[draw] (n9_6) at (12,3.6) {};

            \draw [arrowr] (n1_6) to (n9_6);
            \draw [arrowgd] (n1_6.center) to (n1_5.center);
            \draw [arrowr] (n1_5) to (n9_5);
            \draw [arrowgd] (n1_5.center) to (n1_4.center);
            \draw [arrowg] (n1_4) to (n5_4);
            \draw [arrow] (n1_3) to (n2_3);
            \draw [arrow] (n1_2) to (n2_2);
            \draw [arrow] (n1_1) to (n2_1);

            \draw [arrow] (n2_3) to (n5_3);
            \draw [arrow] (n2_2) to (n5_2);
            \draw [arrow] (n2_1) to (n4_1);

            \draw [arrow] (n4_1) to (n5_1);

            \draw [arrowg] (n5_4) to (n6_4);

            \draw [arrow] (n5_3) to (n6_3);
            \draw [arrow] (n5_2) to (n6_2);
            \draw [arrow] (n5_1) to (n6_1);

            \draw [arrowr] (n6_4) to (n9_4);
            \draw [arrowgd] (n6_4.center) to (n6_3.center);
            \draw [arrowr] (n6_3) to (n8_3);
            \draw [arrowgd] (n6_3.center) to (n6_2.center);
            \draw [arrowr] (n6_2) to (n8_2);
            \draw [arrowgd] (n6_2.center) to (n6_1.center);
            \draw [arrowr] (n6_1) to (n7_1);
            \draw [arrowgd] (n6_1.center) to (n6_0.center);

            \draw [arrow] (n7_1) to (n8_1);

            \draw [arrow] (n8_3) to (n9_3);
            \draw [arrow] (n8_2) to (n9_2);
            \draw [arrow] (n8_1) to (n9_1);

            \draw [arrow] (n1_0) to (n2_0);
            \draw [arrow] (n2_0) to (n3_0);
            \draw [arrow] (n3_0) to (n4_0);
            \draw [arrow] (n4_0) to (n5_0);
            \draw [arrow] (n5_0) to (n6_0);
            \draw [arrow] (n6_0) to (n7_0);
            \draw [arrow] (n7_0) to (n8_0);
            \draw [arrow] (n8_0) to (n9_0);

        \end{tikzpicture}
    \end{figure}

    \newpage
    \subsection*{Laufzeit von Search($x$)}\label{subsec:laufzeitVonSearch}
    \begin{equation*}
        \text{Beweis mit } X_{ik} \text{ möglich: } X_{ik} =
        \[ \begin{cases}
               1 & \text{Falls Turm } i \text{ auf Höhe  } h \text{ besucht} \\
               0 & \text{sonst}
        \end{cases}
        \]
    \end{equation*}

    Besser wäre jedoch, eine Routine zu bauen, die dieselben Zellen besucht wie die Suche, jedoch einfacher zu analysieren ist.

    \begin{algorithm}
        \begin{algorithmic}
            \STATE $v\gets x$
            \STATE $h\gets 0$
            \STATE
            \WHILE{ $ v \neq-\infty$ \AND  $h \neq h_{max}$}
                \IF{$v.\text{height} > h$}
                    \STATE {$h=h+1$}
                \ELSE
                    \STATE{$v=v$.\text{backward[$h$]}}
                \ENDIF
            \ENDWHILE
            \STATE
            \RETURN $v$

        \end{algorithmic}
    \end{algorithm}

    Man muss sich also den \textbf{\textsf{Search($x$)}}-Algorithmus rückwärts vorstellen.
    Damit lässt sich beobachten

    \begin{itemize}
        \item Die Wahrscheinlichkeit, dass sich über der aktuellen Zelle noch eine weitere Zelle befindet, ist $\frac{1}{2}$.
        \item Im \texttt{if-else} - Zweig entscheidet sich, ob hoch (WS $=\frac{1}{2}$) oder nach links (WS $=\frac{1}{2}$) weitergegangen wird.
        \item [$\rightarrow$] Erwartete Anzahl an Linksschritten = Erwartete Anzahl an Rechtsschritten
        \item Für die erwartete Maximalhöhe \textbf{\textsf{eines}} Turmes ergibt sich damit:
        \begin{equation*}
            \begin{flalign}
                & P(h_i \geq h) &= 2^{-h} \\\nonumber
                \rotatebox[origin=c]{180}{$\Lsh$} & P(h_i \geq 2 \cdot \log(n) - 1) &\leq \frac{1}{n^2}
            \end{flalign}
        \end{equation*}
        \item Und für die erwartete Maximalhöhe \textbf{\textsf{aller}} Türme somit:
        \begin{equation*}
            \begin{flalign}
                & P(h_1 \geq 2 \cdot \log(n) \| h_2 \geq 2 \cdot \log(n) \| ... \| h_n \geq 2 \cdot \log(n))\\\nonumber
                \leq &\  \frac{1}{n^2} + \frac{1}{n^2} + \ldots + \frac{1}{n^2}\\\nonumber
                = &\  n\cdot \frac{1}{n^2} = \frac{n}{n^2} = \frac{1}{n}
            \end{flalign}
        \end{equation*}
        \item [$\Rightarrow$] Mit hoher Wahrscheinlichkeit ist \textbf{\textsf{Search($x$)}} also $\mathcal{O}(\log(n))$.
    \end{itemize}

    \newpage
    \subsection*{Visualisierung der Laufzeit von Search($x$) und Analyse mit Master-Theorem}\label{subsec:visualisierung}

    \begin{figure}[H]
        \centering
        \begin{tabular}{|C{2em}|C{2em}|C{2em}|C{2em}|C{2em}|C{1em}|C{2em}|C{2em}|C{2em}|C{2em}|C{2em}|}
            \multicolumn{5}{C{10em}}{Rekursion}& \multicolumn{1}{C{1em}}{} & \multicolumn{5}{C{10em}}{Aufwand} \\
            \multicolumn{5}{C{10em}}{}& \multicolumn{1}{C{1em}}{} & \multicolumn{5}{C{10em}}{} \\\cline{1-5} \cline{7-11}
            \multicolumn{5}{|C{10em}|}{$n$}& & \multicolumn{5}{C{9em}|}{$n^c$} \\ \cline{1-5} \cline{7-11}
            \multicolumn{1}{C{2em}}{}& \multicolumn{1}{C{2em}}{} & \multicolumn{1}{C{2em}}{} & \multicolumn{1}{C{2em}}{} &
            \multicolumn{1}{C{2em}}{} & \multicolumn{1}{C{1em}}{} & \multicolumn{1}{C{2em}}{} & \multicolumn{1}{C{2em}}{} &
            \multicolumn{1}{C{2em}}{} & \multicolumn{1}{C{2em}}{} & \multicolumn{1}{C{2em}}{}\\  \cline{1-1} \cline{3-3} \cline{5-5} \cline{7-7} \cline{9-9} \cline{11-11}
            $\frac{n}{b}$& $+$ &$\frac{n}{b}$& \ldots & $\frac{n}{b}$ & & $\left(\frac{n}{b}\right)^c$ & $+$ & $\left(\frac{n}{b}\right)^c$& \ldots & $\left(\frac{n}{b}\right)^c$ \\ \cline{1-1} \cline{3-3} \cline{5-5} \cline{7-7} \cline{9-9} \cline{11-11}
            \multicolumn{5}{C{10em}}{\underbracetext{\phantom{we must} $a$-mal \phantom{we must eat more}}}& \multicolumn{1}{C{1em}}{}& \multicolumn{5}{C{9em}}{\underbracetext{\phantom{we must} $a$-mal \phantom{we must eat more}}} \\
            \multicolumn{1}{C{2em}}{}& \multicolumn{1}{C{2em}}{} & \multicolumn{1}{C{2em}}{} & \multicolumn{1}{C{2em}}{} &
            \multicolumn{1}{C{2em}}{} & \multicolumn{1}{C{1em}}{} & \multicolumn{1}{C{2em}}{} & \multicolumn{1}{C{2em}}{} &
            \multicolumn{1}{C{2em}}{} & \multicolumn{1}{C{2em}}{} & \multicolumn{1}{C{2em}}{}\\  \cline{1-1} \cline{3-3} \cline{5-5} \cline{7-7} \cline{9-9} \cline{11-11}
            $\frac{n}{b^2}$& $+$ &$\frac{n}{b^2}$& \ldots & $\frac{n}{b^2}$ & & $\left(\frac{n}{b^2}\right)^c$ & $+$ & $\left(\frac{n}{b^2}\right)^c$& \ldots & $\left(\frac{n}{b^2}\right)^c$ \\ \cline{1-1} \cline{3-3} \cline{5-5} \cline{7-7} \cline{9-9} \cline{11-11}
            \multicolumn{5}{C{10em}}{\underbracetext{\phantom{we must} $a^2$-mal \phantom{we must eat more}}}& \multicolumn{1}{C{1em}}{}& \multicolumn{5}{C{9em}}{\underbracetext{\phantom{we must} $a^2$-mal \phantom{we must eat more}}} \\
            \multicolumn{1}{C{2em}}{}& \multicolumn{1}{C{2em}}{} & \multicolumn{1}{C{2em}}{\vdots} & \multicolumn{1}{C{2em}}{} &
            \multicolumn{1}{C{2em}}{} & \multicolumn{1}{C{1em}}{} & \multicolumn{1}{C{2em}}{} & \multicolumn{1}{C{2em}}{} &
            \multicolumn{1}{C{2em}}{\vdots} & \multicolumn{1}{C{2em}}{} & \multicolumn{1}{C{2em}}{}\\
        \end{tabular}
    \end{figure}

    \begin{equation*}
        \sum^{\log_b(n)}_{l=0} n^c \cdot \underbrace{\left( \frac{a}{b^c}\right)^l}_{q} = n^c \cdot \sum^{\log_b(n)}_{l=0} q^l \quad \text{(Geom. Reihe)}
    \end{equation*}
    \begin{equation*}
        \text{Also }
        \[ \begin{cases}
               \leq & n^c \cdot \frac{1}{1-q} \in \Theta(n^c), \qquad q < 1 \Leftrightarrow \log_b(a) < c \\
               = & n^c \cdot log_b(n) \in \Theta(n^c \cdot \log(n)), \qquad q = 1 \Leftrightarrow \log_b(a) = c \\
               > &\in \Theta(n^{\log_b(a)}), \qquad q > 1 \Leftrightarrow \log_b(a) > c
        \end{cases}
        \]
    \end{equation*}

    Quick-Sort mit $a=2$, $b=2$ und $c=1$ entspricht also der zweiten Klasse: $\log_2(2)=1=c$

    \hrulefill

    \section*{Anhang}\label{sec:anhang}
    \subsubsection*{Master-Theorem}
    Rekursionsgleichungen für \textit{Divide} \& \textit{Conquer}-Algorithmen lassen sich mit dem Master-Theorem lösen:
    \begin{equation*}
        \begin{flalign}
            & T(n) = a \cdot T(\frac{n}{b}) + \Theta(n^c)&  \\\nonumber
            a\ & \tilde{=}\ \text{Anzahl der Teilprobleme} & |\ a \geq 1\\\nonumber
            b\ & \tilde{=}\ \text{Verkleinerung des Teilproblems} & |\ b > 1\\\nonumber
            c\ & \tilde{=}\ \text{Aufwand für \textit{Divide} \& \textit{Merge}} & |\ c \geq 1
        \end{flalign}
    \end{equation*}

\end{document}
