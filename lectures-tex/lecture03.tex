\documentclass{scrartcl}% siehe <http://www.komascript.de>
% packages ---------------------------------------------------------------------------------------------------- packages
\usepackage[margin=3cm]{geometry}
\usepackage[utf8]{inputenc}
\usepackage[T1]{fontenc}
\usepackage[ngerman]{babel}
\usepackage{lmodern}
\usepackage{amsmath}
\usepackage{amssymb}
\usepackage{amsfonts}
\usepackage{cancel}
\usepackage{listings}
\usepackage{multirow}
\usepackage{hhline}
\usepackage{tikz}
\usepackage{pgfplots}
\usepackage[all,dvips,arc,curve,color,frame]{xy}
\usepackage[yyyymmdd,hhmm]{datetime}
\usepackage{float}
\usepackage{hyperref}
\usepackage[table]{xcolor,colortbl}
\usepackage{graphicx}
\usepackage{subcaption}
\usepackage{tabularx}
\usepackage{footnote}
\usepackage{multicol}
\usepackage{stackengine}
\usepackage[norndcorners,customcolors,shade]{hf-tikz}
\usepackage{siunitx}
\usepackage{algorithm}
\usepackage[]{algorithmic}
\usepackage{centernot}
\usepackage{amsthm}
% \usepackage[noend]{algorithmic}

% pgf plotsets -------------------------------------------------------------------------------------------- pgf plotsets
\pgfplotsset{ticks=none}
\pgfplotsset{width=10cm, compat=1.5.1}

% tikz libraries ---------------------------------------------------------------------------------------- tikz libraries
\usetikzlibrary{shapes}
\usetikzlibrary{positioning}
\usetikzlibrary{shapes.misc}
\usetikzlibrary{shapes.arrows,calc,quotes,babel}
\usetikzlibrary{positioning,arrows,chains,scopes,fit}
\usetikzlibrary{matrix,backgrounds}
\usetikzlibrary{trees}
\usetikzlibrary{calc}
\usetikzlibrary{tikzmark}
\usetikzlibrary{intersections}

\sisetup{
locale = DE ,
per-mode = symbol
}

\algsetup{indent=2em}
\newlength\myindent
\setlength\myindent{2em}
\newcommand\bindent{%
    \begingroup
    \setlength{\itemindent}{\myindent}
    \addtolength{\algorithmicindent}{\myindent}
}
\newcommand\eindent{\endgroup}

% commands ---------------------------------------------------------------------------------------------------- commands
\setlength{\parindent}{0pt}
\newcommand{\includepic}[2]{\includegraphics[width=#2\textwidth,height=#2\textheight,keepaspectratio]{#1}}
\newcommand\circlearound[1]{%
\tikz[baseline]\node[draw,shape=circle,anchor=base] {#1} ;}
\newcommand{\emptyrow}[0]{\multicolumn{1}{c}{} & \multicolumn{1}{c}{} & \multicolumn{1}{c}{} & \multicolumn{1}{c}{} &
\multicolumn{1}{c}{} & \multicolumn{1}{c}{} & \multicolumn{1}{c}{} & \multicolumn{1}{c}{} &
\multicolumn{1}{c}{} & \multicolumn{1}{c|}{} & & \\}
\newcommand\underbracetext[1]{\raisebox{1ex}{\ensuremath{\underbrace{\hphantom{\text{#1}}}_{\text{\normalsize #1}}}}}
\newcommand{\headerline}[3]{
\subject{#2}
\title{#1}
\subtitle{#3}
\date{Letztes Update: \today \ - \currenttime \ Uhr}
\maketitle
}
\newcommand{\header}[3]{\section*{#1: #3 - #2}\subsection*{Letztes Update: \today \ - \currenttime \ Uhr}\vspace*{1cm}}
\newcommand{\newproof}[2]{
\vspace*{0.3cm} \textbf{\textsf{Satz:}} #1

\vspace*{0.3cm} \textbf{\textsf{Beweis:}} #2 \vspace*{0.3cm}
}
%\newcommand{\uparrow}[0]{\rlap{\scalebox{1.6}{$\uparrow$}}}
\newlength\dlf
\newcommand{\proofend}[0]{\hfill $\square$}
\newcommand\drawbi[7][4pt]{%
    \path(#2)--(#3)coordinate[at start](h1)coordinate[at end](h2);
    \draw[#4]($(h1)!#1!90:(h2)$)-- node [auto=left] {#5} ($(h2)!#1!-90:(h1)$);
    \draw[#6]($(h1)!#1!-90:(h2)$)-- node [auto=right] {#7} ($(h2)!#1!90:(h1)$);
    }

% colors -------------------------------------------------------------------------------------------------------- colors
\definecolor{amethyst}{rgb}{0.6, 0.4, 0.8}
\definecolor{anti-flashwhite}{rgb}{0.95, 0.95, 0.96}
\definecolor{applegreen}{rgb}{0.55, 0.71, 0.0}
\definecolor{babyblue}{rgb}{0.54, 0.81, 0.94}
\definecolor{bistre}{rgb}{0.24, 0.17, 0.12}
\definecolor{buff}{rgb}{0.94, 0.86, 0.51}
\definecolor{dodgerblue}{rgb}{0.12, 0.56, 1.0}
\definecolor{icterine}{rgb}{0.99, 0.97, 0.37}
\definecolor{tan}{rgb}{0.82, 0.71, 0.55}
\definecolor{upforestgreen}{rgb}{0.0, 0.27, 0.13}

% listing colors
\definecolor{listingBlue}{rgb}{0.13,0.13,1}
\definecolor{listingGreen}{rgb}{0,0.5,0}
\definecolor{listingRed}{rgb}{0.9,0,0}
\definecolor{listingGrey}{rgb}{0.46,0.45,0.48}

% listings ---------------------------------------------------------------------------------------------------- listings
\lstset{basicstyle=\ttfamily}
\lstset{literate=%
{Ö}{{\"O}}1
{Ä}{{\"A}}1
{Ü}{{\"U}}1
{ß}{{\ss}}1
{ü}{{\"u}}1
{ä}{{\"a}}1
{ö}{{\"o}}1
}
\lstset{language=Java,
showspaces=false,
showtabs=false,
breaklines=true,
showstringspaces=false,
breakatwhitespace=true,
commentstyle=\color{listingGreen},
keywordstyle=\color{listingBlue},
stringstyle=\color{listingRed},
basicstyle=\ttfamily,
moredelim=[il][\textcolor{listingGrey}]{$$},
moredelim=[is][\textcolor{listingGrey}]{\%\%}{\%\%}
}
\lstdefinelanguage{Kotlin}{
  keywords={package, as, typealias, this, super, val, var, fun, for, null, true, false, is, in, throw, return, break, continue, object, if, try, else, while, do, when, yield, typeof, yield, typeof, class, interface, enum, object, override, public, private, get, set, import, abstract, procedure},
  keywordstyle=\color{listingBlue}\bfseries,
  ndkeywords={@Deprecated, Iterable, Int, Integer, Float, Double, String, Runnable, dynamic},
  ndkeywordstyle=\color{amethyst}\bfseries,
  emph={println, return@, forEach,},
  emphstyle={\color{orange}},
  identifierstyle=\color{black},
  sensitive=true,
  commentstyle=\color{listingGreen}\ttfamily,
  comment=[l]{//},
  morecomment=[s]{/*}{*/},
  stringstyle=\color{listingRed}\ttfamily,
  morestring=[b]",
  morestring=[s]{"""*}{*"""},
}

% misc ------------------------------------------------------------------------------------------------------------ misc
\graphicspath{ {../lectures-img/} }
\setlength{\parindent}{0pt}

\newcolumntype{C}[1]{>{\centering\arraybackslash}p{#1}} % zentriert mit Breitenangabe
\newcolumntype{L}[1]{>{\raggedright\arraybackslash}p{#1}} % rechtsbündig mit Breitenangabe
\newcolumntype{R}[1]{>{\raggedleft\arraybackslash}p{#1}} % rechtsbündig mit Breitenangabe

% text ------------------------------------------------------------------------------------------------------------ text
\newcommand{\fett}[1]{\textbf{\textsf{#1}}}
\newcommand{\kursiv}[1]{\textit{#1}}

% proofs -------------------------------------------------------------------------------------------------------- proofs
\newtheoremstyle{dotless}{}{}{\itshape}{}{\bfseries\sffamily}{}{ }{}
\theoremstyle{dotless}
\newtheorem*{satz}{Satz:}
\renewenvironment{proof}{{\bfseries \sffamily Beweis:}}{\hfill $\square$}


\begin{document}
    \header{Algorithmen und Berechenbarkeit}{Mitschrift}{Vorlesung 03}

    \section*{Landau-Symbole und Laufzeiten}
    \label{subsec:laufzeiten}

    \begin{itemize}
        \item $\mathcal{O}(n^2)$ Die Menge aller Funktionen, die (asymptotisch)\\
        nicht schneller wachsen als $n^2.\hfill (\leq)$
        \item $o(n^2)$ Die Menge aller Funktionen, die (asymptotisch)\\
        echt langsamer wachsen als $n^2.\hfill (<)$
        \item $\Omega(n^2)$ Die Menge aller Funktionen, die (asymptotisch)\\
        mindestens so schnell wachsen wie $n^2.\hfill (\geq)$
        \item $\omega(n^2)$ Die Menge aller Funktionen, die (asymptotisch)\\
        echt schneller wachsen als $n^2.\hfill (>)$
        \item $\Theta(n^2)$ Die Menge aller Funktionen, die (asymptotisch)\\
        genau so schnell wachsen wie $n^2.\hfill (=)$
    \end{itemize}

    \section*{Einschub: Vergleichsbasiertes Sortieren}
    \label{subsec:vergleichsbasiertessortieren}

    Beim \kursiv{vergleichsbasierten Sortieren} von $n$ Objekten,
    wobei die Elemente nur verglichen werden dürfen,
    liefert jeder Algorithmus eine obere Schranke für $T(n)$ (Laufzeit für $n$ Elemente).
    Zum Beispiel ist die obere Schranke von \fett{Bubblesort} $\in \mathcal{O}(n^2)$
    und die von \fett{Mergesort} $\in \mathcal{O}(n \cdot \log(n))$.

    \begin{satz}
        Es kann bewiesen werden, dass $T(n) \in \Omega(n \cdot \log(n)) \Rightarrow T(n) \in \Theta(n \cdot \log(n))$.
    \end{satz}

    \begin{proof}
        Sei $A$ ein beliebiger Algorithmus zum Sortieren. $A$ vergleicht $e_i$ mit $e_j$.
        Die Vergleiche können exemplarisch als Binärbaum skizziert werden.
        \begin{figure}[htb]
            \centering
            \begin{tikzpicture}[level distance=1.5cm,
            level 1/.style={sibling distance=7cm},
            level 2/.style={sibling distance=4.5cm},
            every node/.style = {shape=rectangle, rounded corners,
            draw, align=center,
            top color=upforestgreen!30, bottom color=upforestgreen!30}]
                \node {Vergleiche $e_1$ mit $e_2$}
                child { node {Vergleiche $e_3$ mit $e_5$}child {node {\dots}}
                child {node {\dots}}
                }
                child { node {Vergleiche $e_4$ mit $e_6$}child {node {\dots}}
                child {node {\dots}}
                }
                ;
            \end{tikzpicture}
        \end{figure}

        Trifft der Algorithmus auf ein Blatt des Binärbaums, so hat der Algorithmus fertig sortiert,
        bzw.\ der Algorithmus hat \kursiv{herausgefunden}, was die Permutation der Eingabe war.
        Es lässt sich nun festhalten

        \begin{enumerate}
            \item Der Baum muss $n!$ Blätter haben.
            \item Die Worst-Case-Laufzeit des Algorithmus $A$ entspricht genau der Tiefe des Baums.
        \end{enumerate}

        Nun lässt sich die minimale Tiefe eines Binärbaums, der $n!$ Blätter hat, berechnen. Sei

        \begin{equation*}
            2^n \approx n! \approx \left({\frac{n}{e}}\right)^{\frac{n}{e}}
        \end{equation*}

        Sei $x$ die minimale Tiefe. Dann gilt

        \begin{align*}
            x &= \log_2\left(\frac{n}{e}\right)^{\frac{n}{e}}\\
            &= n \cdot \log(n)
        \end{align*}

    \end{proof}

    \section*{Randomisierte Algorithmen: Monte-Carlo und Las-Vegas}
    In Vorlesung 01 und 02 wurden zwei verschiedene Arten von randomisierten Algorithmen angeschnitten:
    Las-Vegas- bzw.\ Monte-Carlo-Algorithmen.

    \begin{itemize}
        \item \fett{Las-Vegas:}
        \begin{itemize}
            \item \kursiv{Charakterisierung:} Die Laufzeit hängt vom Zufall ab, die Korrektheit nicht.
            \item \kursiv{Beispiel}: Randomisierter Closest-Pair-Algorithmus
            \item \kursiv{Umwandlung in Monte-Carlo-Algorithmus:}
            Die Umwandlung von Las-Vegas- in Monte-Carlo-Algorithmen ist möglich.
            Sei $A_L$ ein Las-Vegas-Algorithmus.
            Man lässt $A$ eine bestimmte Anzahl an Schritten laufen und bricht dann ab.
            War $A$ bis dahin fertig, so muss das Ergebnis korrekt sein.
            War $A$ bis dahin nicht fertig, dann gibt es auch kein korrektes Ergebnis.

            Sei $f(n)$ die erwartete Laufzeit von $A$. $A$ darf nun für maximal $\alpha \cdot f(n)$
            (mit $\alpha \geq 1$) Schritte laufen.
            Es gilt: $A$ hat immer eine Laufzeit von $< \alpha \cdot f(n)$.
            Die Wahrscheinlichkeit, dass $A$ ein falsches Resultat liefert,
            entspricht genau der Wahrscheinlichkeit, dass $A$ länger als $\alpha \cdot f(n)$ Zeit benötigt.
        \end{itemize}
        \item \fett{Monte-Carlo:}
        \begin{itemize}
            \item \kursiv{Charakterisierung:} Die Laufzeit hängt nicht vom Zufall ab, die Korrektheit schon.
            \item \kursiv{Beispiel}: Kargers Min-Cut-Algorithmus
            \item \kursiv{Umwandlung in Las-Vegas-Algorithmus:}
            Die Umwandlung ist nicht für alle Monte-Carlo-Algorithmen können ohne Weiteres möglich.
            Für manche Probleme ist die Verifikation des Ergebnisses einfacher als die Berechnung:
            \begin{itemize}
                \item Sortieren: $\mathcal{O}(n \cdot \log(n))$ vs.\ $O(n)$
                \item Kürzeste Wege: $\mathcal{O}(n \cdot \log(n+m))$ vs.\ $O(m)$
            \end{itemize}
            Das Überführen von Monte-Carlo- in Las-Vegas-Algorithmen ist möglich,
            wenn ein effizienter \textit{Checker} existiert. Sei $B$ ein Monte-Carlo-Algorithmus mit Laufzeit $f(n)$,
            sei $C$ ein Checker mit Laufzeit $f(n)$, die Erfolgswahrscheinlichkeit von $B$ sei $p(n)$.
            \newpage
            Ein Las-Vegas-Algorithmus kann nun auf folgende Weise erzeugt werden:
            \begin{figure}[H]
                \centering
                \begin{minipage}{.55\linewidth}
                    \begin{algorithmic}
                        \STATE{lasse $B$ laufen}
                        \IF {$B.result$ ist korrekt}
                        \RETURN {$B.result$}
                        \ELSE
                        \STATE {starte neu}
                        \ENDIF
                    \end{algorithmic}
                \end{minipage}
            \end{figure}

            Für die erwartete Laufzeit $R_B$ ergibt sich
            \begin{align*}
                R_B &= p(n) \cdot \left( f(n) + g(n) \right) \\
                &\quad + ((1 - p(n)) \cdot p(n) \cdot (f(n) + g(n))) \cdot 2\\
                &\quad + ((1 - p(n))^2 \cdot p(n) \cdot (f(n) + g(n))) \cdot 3 +\ \dots \\\nonumber
                &=(f(n) + g(n)) \cdot p(n) \cdot \sum^{\infty}_{i=1}\left( 1-p(n)\right)^{i}\\
                & < \frac{f(n) + g(n)}{p(n)} \cdot (i+1)
            \end{align*}
        \end{itemize}
    \end{itemize}

    \section*{Anhang}
    \label{sec:anhang}

    \subsubsection*{Markov-Ungleichung}
    Sei $X$ eine nicht negative Zufallsvariable mit $E[X] = \mu$.
    Dann gilt:

    \begin{equation*}
        P(X \geq \alpha \cdot \mu ) \leq \frac{1}{\alpha}
    \end{equation*}

\end{document}