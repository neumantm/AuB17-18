\documentclass{scrartcl}% siehe <http://www.komascript.de>
% packages ---------------------------------------------------------------------------------------------------- packages
\usepackage[margin=3cm]{geometry}
\usepackage[utf8]{inputenc}
\usepackage[T1]{fontenc}
\usepackage[ngerman]{babel}
\usepackage{lmodern}
\usepackage{amsmath}
\usepackage{amssymb}
\usepackage{amsfonts}
\usepackage{cancel}
\usepackage{listings}
\usepackage{multirow}
\usepackage{hhline}
\usepackage{tikz}
\usepackage{pgfplots}
\usepackage[all,dvips,arc,curve,color,frame]{xy}
\usepackage[yyyymmdd,hhmm]{datetime}
\usepackage{float}
\usepackage{hyperref}
\usepackage[table]{xcolor,colortbl}
\usepackage{graphicx}
\usepackage{subcaption}
\usepackage{tabularx}
\usepackage{footnote}
\usepackage{multicol}
\usepackage{stackengine}
\usepackage[norndcorners,customcolors,shade]{hf-tikz}
\usepackage{siunitx}
\usepackage{algorithm}
\usepackage[]{algorithmic}
\usepackage{centernot}
\usepackage{amsthm}
% \usepackage[noend]{algorithmic}

% pgf plotsets -------------------------------------------------------------------------------------------- pgf plotsets
\pgfplotsset{ticks=none}
\pgfplotsset{width=10cm, compat=1.5.1}

% tikz libraries ---------------------------------------------------------------------------------------- tikz libraries
\usetikzlibrary{shapes}
\usetikzlibrary{positioning}
\usetikzlibrary{shapes.misc}
\usetikzlibrary{shapes.arrows,calc,quotes,babel}
\usetikzlibrary{positioning,arrows,chains,scopes,fit}
\usetikzlibrary{matrix,backgrounds}
\usetikzlibrary{trees}
\usetikzlibrary{calc}
\usetikzlibrary{tikzmark}
\usetikzlibrary{intersections}

\sisetup{
locale = DE ,
per-mode = symbol
}

\algsetup{indent=2em}
\newlength\myindent
\setlength\myindent{2em}
\newcommand\bindent{%
    \begingroup
    \setlength{\itemindent}{\myindent}
    \addtolength{\algorithmicindent}{\myindent}
}
\newcommand\eindent{\endgroup}

% commands ---------------------------------------------------------------------------------------------------- commands
\setlength{\parindent}{0pt}
\newcommand{\includepic}[2]{\includegraphics[width=#2\textwidth,height=#2\textheight,keepaspectratio]{#1}}
\newcommand\circlearound[1]{%
\tikz[baseline]\node[draw,shape=circle,anchor=base] {#1} ;}
\newcommand{\emptyrow}[0]{\multicolumn{1}{c}{} & \multicolumn{1}{c}{} & \multicolumn{1}{c}{} & \multicolumn{1}{c}{} &
\multicolumn{1}{c}{} & \multicolumn{1}{c}{} & \multicolumn{1}{c}{} & \multicolumn{1}{c}{} &
\multicolumn{1}{c}{} & \multicolumn{1}{c|}{} & & \\}
\newcommand\underbracetext[1]{\raisebox{1ex}{\ensuremath{\underbrace{\hphantom{\text{#1}}}_{\text{\normalsize #1}}}}}
\newcommand{\headerline}[3]{
\subject{#2}
\title{#1}
\subtitle{#3}
\date{Letztes Update: \today \ - \currenttime \ Uhr}
\maketitle
}
\newcommand{\header}[3]{\section*{#1: #3 - #2}\subsection*{Letztes Update: \today \ - \currenttime \ Uhr}\vspace*{1cm}}
\newcommand{\newproof}[2]{
\vspace*{0.3cm} \textbf{\textsf{Satz:}} #1

\vspace*{0.3cm} \textbf{\textsf{Beweis:}} #2 \vspace*{0.3cm}
}
%\newcommand{\uparrow}[0]{\rlap{\scalebox{1.6}{$\uparrow$}}}
\newlength\dlf
\newcommand{\proofend}[0]{\hfill $\square$}
\newcommand\drawbi[7][4pt]{%
    \path(#2)--(#3)coordinate[at start](h1)coordinate[at end](h2);
    \draw[#4]($(h1)!#1!90:(h2)$)-- node [auto=left] {#5} ($(h2)!#1!-90:(h1)$);
    \draw[#6]($(h1)!#1!-90:(h2)$)-- node [auto=right] {#7} ($(h2)!#1!90:(h1)$);
    }

% colors -------------------------------------------------------------------------------------------------------- colors
\definecolor{amethyst}{rgb}{0.6, 0.4, 0.8}
\definecolor{anti-flashwhite}{rgb}{0.95, 0.95, 0.96}
\definecolor{applegreen}{rgb}{0.55, 0.71, 0.0}
\definecolor{babyblue}{rgb}{0.54, 0.81, 0.94}
\definecolor{bistre}{rgb}{0.24, 0.17, 0.12}
\definecolor{buff}{rgb}{0.94, 0.86, 0.51}
\definecolor{dodgerblue}{rgb}{0.12, 0.56, 1.0}
\definecolor{icterine}{rgb}{0.99, 0.97, 0.37}
\definecolor{tan}{rgb}{0.82, 0.71, 0.55}
\definecolor{upforestgreen}{rgb}{0.0, 0.27, 0.13}

% listing colors
\definecolor{listingBlue}{rgb}{0.13,0.13,1}
\definecolor{listingGreen}{rgb}{0,0.5,0}
\definecolor{listingRed}{rgb}{0.9,0,0}
\definecolor{listingGrey}{rgb}{0.46,0.45,0.48}

% listings ---------------------------------------------------------------------------------------------------- listings
\lstset{basicstyle=\ttfamily}
\lstset{literate=%
{Ö}{{\"O}}1
{Ä}{{\"A}}1
{Ü}{{\"U}}1
{ß}{{\ss}}1
{ü}{{\"u}}1
{ä}{{\"a}}1
{ö}{{\"o}}1
}
\lstset{language=Java,
showspaces=false,
showtabs=false,
breaklines=true,
showstringspaces=false,
breakatwhitespace=true,
commentstyle=\color{listingGreen},
keywordstyle=\color{listingBlue},
stringstyle=\color{listingRed},
basicstyle=\ttfamily,
moredelim=[il][\textcolor{listingGrey}]{$$},
moredelim=[is][\textcolor{listingGrey}]{\%\%}{\%\%}
}
\lstdefinelanguage{Kotlin}{
  keywords={package, as, typealias, this, super, val, var, fun, for, null, true, false, is, in, throw, return, break, continue, object, if, try, else, while, do, when, yield, typeof, yield, typeof, class, interface, enum, object, override, public, private, get, set, import, abstract, procedure},
  keywordstyle=\color{listingBlue}\bfseries,
  ndkeywords={@Deprecated, Iterable, Int, Integer, Float, Double, String, Runnable, dynamic},
  ndkeywordstyle=\color{amethyst}\bfseries,
  emph={println, return@, forEach,},
  emphstyle={\color{orange}},
  identifierstyle=\color{black},
  sensitive=true,
  commentstyle=\color{listingGreen}\ttfamily,
  comment=[l]{//},
  morecomment=[s]{/*}{*/},
  stringstyle=\color{listingRed}\ttfamily,
  morestring=[b]",
  morestring=[s]{"""*}{*"""},
}

% misc ------------------------------------------------------------------------------------------------------------ misc
\graphicspath{ {../lectures-img/} }
\setlength{\parindent}{0pt}

\newcolumntype{C}[1]{>{\centering\arraybackslash}p{#1}} % zentriert mit Breitenangabe
\newcolumntype{L}[1]{>{\raggedright\arraybackslash}p{#1}} % rechtsbündig mit Breitenangabe
\newcolumntype{R}[1]{>{\raggedleft\arraybackslash}p{#1}} % rechtsbündig mit Breitenangabe

% text ------------------------------------------------------------------------------------------------------------ text
\newcommand{\fett}[1]{\textbf{\textsf{#1}}}
\newcommand{\kursiv}[1]{\textit{#1}}

% proofs -------------------------------------------------------------------------------------------------------- proofs
\newtheoremstyle{dotless}{}{}{\itshape}{}{\bfseries\sffamily}{}{ }{}
\theoremstyle{dotless}
\newtheorem*{satz}{Satz:}
\renewenvironment{proof}{{\bfseries \sffamily Beweis:}}{\hfill $\square$}


\usepackage{diagbox}

\definecolor{cadmiumred}{rgb}{0.89, 0.0, 0.13}
\definecolor{cadmiumorange}{rgb}{0.93, 0.53, 0.18}
\definecolor{cadmiumyellow}{rgb}{1.0, 0.96, 0.0}
\definecolor{cadmiumgreen}{rgb}{0.0, 0.42, 0.24}
\definecolor{blizzardblue}{rgb}{0.67, 0.9, 0.93}
\definecolor{persianblue}{rgb}{0.11, 0.22, 0.73}
\definecolor{darkviolet}{rgb}{0.58, 0.0, 0.83}

\begin{document}
    \headerline{Algorithmen und Berechenbarkeit}{Vorlesungsmitschrift}{Vorlesung 17}

    \section*{Die Komplexitätsklasse $\mathcal{N}\mathcal{P}$}
    Ein Problem $\mathcal{X}$ ist in der Komplexitätsklasse $\mathcal{P}$,
    wenn es einen Polynomzeitalgorithmus für $\mathcal{X}$ gibt
    (\textit{alternativ:} Ein Problem $\mathcal{X}$ ist in der Komplexitätsklasse $\mathcal{P}$,
    wenn es eine TM $\mathcal{M}$ gibt, die $\mathcal{X}$ in einer polynomiellen Anzahl an Schritten löst).

    \vspace*{0.3cm}
    \textbf{\textsf{Definition Akzeptanzverhalten einer NTM:}}
    Eine NTM $\mathcal{M}$ akzeptiert eine Eingabe $x \in \Sigma^*$
    falls es mindestens eine Sequenz von gültigen Rechenschritten (gemäß Übergangsrelation) gibt,
    die in einer akzeptierenden Konfiguration endet.

    \vspace*{0.3cm}
    \textbf{\textsf{Definition Laufzeit einer NTM:}}
    Sei $\mathcal{M}$ eine NTM. Die Laufzeit $T_{\mathcal{M}}(x)$ von $\mathcal{M}$ auf einer Eingabe $x \in L(\mathcal{M})$ ist definiert als
    \begin{align*}
        T_{\mathcal{M}}(x) := & \text{ Länge des kürzesten akzeptierenden Rechenwegs von } \mathcal{M} \text{ auf } x
    \end{align*}

    Außerdem gilt: Für ein $x \notin L(\mathcal{M})$ ist $T_{\mathcal{M}}(x) = 0$. Die Worst-Case-Laufzeit
    $t_{\mathcal{M}}(n)$ für $\mathcal{M}$ auf Eingaben der Länge $n$ ist
    \begin{align*}
        t_{\mathcal{M}}(n) := &\ max\{T_{\mathcal{M}}(x)\ |\ x\in \Sigma^n \}
    \end{align*}

    \vspace*{0.3cm}
    \textbf{\textsf{Definition Komplexitätsklasse $\mathcal{N}\mathcal{P}$:}}
    $\mathcal{N}\mathcal{P}$ ist die Klasse der Entscheidungsprobleme, die durch eine NTM $\mathcal{M}$ erkannt wird,
    deren Worst-Case-Laufzeit $t_{\mathcal{M}}(n)$ polynomiell \textbf{in $n$} beschränkt ist.

    $\mathcal{N}\mathcal{P}$ bedeutet \textit{Nichtdeterministisch Polynomiell}.

    \subsection*{Beispiel für ein $\mathcal{N}\mathcal{P}$-Problem: CLIQUE}
    Das \textit{CLIQUE}-Problem liegt nicht in $\mathcal{P}$ und ist definiert wie folgt:
    Gegeben sei ein ungerichteter Graph $G(V,E)$ und ein $k \in {\{1, \dots, |V|\}}$.
    Nun möchte man wissen, ob $G$ eine CLIQUE der Größe $k$ hat.
    Dieses Problem kann naiv in $\mathcal{O}(n^k)$ entschieden werden, was jedoch nicht polynomiell ist
    (Eine CLIQUE ist dabei eine Teilmenge von Knoten von $G$, die vollständig untereinander verbunden sind).

    \newpage
    \vspace*{0.3cm}
    \textbf{\textsf{Satz:}}
    CLIQUE $\in \mathcal{N}\mathcal{P}$

    \vspace*{0.3cm}
    \textbf{\textsf{Beweis:}} Es wird eine NTM $\mathcal{M}$ beschrieben mit $L(\mathcal{M}) =$CLIQUE,
    die polynomielle Laufzeit hat und wie folgt vorgeht:
    \begin{enumerate}
        \item Falls die Eingabe nicht der Form $(G,K)$ entspricht, wird verworfen.
        \item Sei nun $G=(V, E)$, $N=\text{ Anzahl der Knoten ohne Beschränkung der Allgemeinheit}$ und $V=\{1, \dots, N\}$.

        $\mathcal{M}$ schreibt hinter die Eingabe den String  $\#^N$, der Kopf bewegt sich über das erste $\#$.
        \item $\mathcal{M}$ läuft von links nach rechts über $\#^N$ und ersetzt nichtdeterministisch jedes $\#$ durch $0$ oder $1$.
        Der resultierende String sei $y = (y_1, y_2, \dots, y_N) \in \{0,1\}^N$
        \item Sei $C = \{ i \in V\ |q_i = 1\}$. $\mathcal{M}$ akzeptiert, falls $C = \text{ K-CLIQUE}$.
    \end{enumerate}

    Task $1,2$ und $4$ sind deterministisch, Task $3$ ist nichtdeterministisch.
    Alle Tasks benötigen eine polynomielle Anzahl an Schritten.

    \vspace*{0.3cm}
    Nun muss gezeigt werden, dass $L(\mathcal{M}) = \text{CLIQUE}$.

    \begin{itemize}
        \item [$a)$] Angenommen, $G$ enthält CLIQUE:
        \begin{itemize}
            \item [$\Rightarrow$] Dann existiert mindestens ein $y$, dass zur Akzeptanz in Task $4$ führt
            \item [$\Rightarrow$] Dieses $y$ wird in Task $3$ nichtdeterministisch gefunden
            \item [$\Rightarrow$] $\mathcal{M}$ akzeptiert die Eingabe
        \end{itemize}
        \item [$b)$] Angenommen, $G$ enthält CLIQUE nicht:
        \begin{itemize}
            \item [$\Rightarrow$] Egal was das Ergebnis aus Task $3$ ist, Task $4$ führt nie zur Akzeptanz
        \end{itemize}
        \item [$\Rightarrow$] CLIQUE $\in \mathcal{N}\mathcal{P}$.\proofend
    \end{itemize}

    \textit{(Im Skript Kapitel 3.2.2 lesen)}

    \subsection*{Beispiel für ein $\mathcal{N}\mathcal{P}$-Hart-Problem: Rucksack/Knapsack}
    \subsubsection*{Optimierungsvariante}
    Gegeben sind $n$ Gegenstände $U=\{u_1,u_2,\dots,u_n\}$ mit den jeweiligen
    Werten $w_i$, den Gewichten $g_i$ und einer Rucksackkapazität $G$.
    Nun wird $\mathcal{I} \subseteq {1, \dots, n}$ gewählt, sodass
    \begin{equation*}
        \sum_{i \in \mathcal{I}} g_i \leq G \text{\textbf{\textsf{ und }}} \sum_{i \in \mathcal{I}} w_i \text{ maximal}
    \end{equation*}

    \begin{figure}[H]
        \centering
        \begin{table}[H]
            \centering
            \begin{tabular}{C{1cm}|C{0.5cm}C{0.5cm}C{0.5cm}C{0.5cm}C{0.5cm}C{0.5cm}C{0.5cm}}
               & 1 & 2 & 3 & $\dots$ & $i$ & $\dots$ & n \\ \hline
               1 &  &  &  &  &  &  &  \\
               2 &  &  &  &  &  &  &  \\
               3 &  &  &  &  &  &  &  \\
               $\vdots$ &  &  &  &  &  &  &  \\
               $i$ &  &  &  &  &  & $\square_{i,j}$ &  \\
               $\vdots$ &  &  &  &  &  &  &  \\
               $\sum_{w_i}$ &  &  &  &  &  &  &  \\

            \end{tabular}
        \end{table}
    \end{figure}

    $\square_{i,j}$ bezeichnet dabei das minimale Gewicht eines Rucksacks mit der Auswahl aus $\{1, \dots, i\}$ und Wert genau $j$.

    \vspace*{0.3cm}
    Betrachtet man nun einen Rucksackinhalt mit Wert $j$ und Gewicht $\square_{i,j}$:

    \begin{itemize}
        \item [$a)$] Falls der Gegenstand $i$ enthalten ist, so hat der Rucksack ein Gewicht von $g_{i} + \square_{i-1,j-1}$.
        \item [$b)$] Falls der Gegenstand $i$ nicht enthalten ist, so gilt $\square_{i,j} = \square_{i-1, j}$
    \end{itemize}

    Insgesamt also
    \begin{equation*}
        \square_{i,j} = \text{ min}(\square_{i-1,j}, g_i + \square_{i-1,j-w_i})
    \end{equation*}

    Als Laufzeit ergibt sich $\mathcal{O}(n \cdot \sum w_i)$.
    Das ist \textit{pseudopolynomiell}, da die Laufzeit sehr stark von den kodierten Zahlen abhängt.
    Eine vergleichsweise kleine Vergrößerung der Eingabe kann die Laufzeit extrem stark ansteigen lassen.

    \subsubsection*{Entscheidungsvariante}
    Gegeben sind wieder $U, g_i, G \in \mathbb{N}, w_i, W \in \mathbb{N}$. Man stellt sich nun die Frage: Existiert ein $K \subseteq U$, für das gilt:
    \begin{equation*}
        \sum_{u_i \in K} g_i \leq G \text{\textbf{\textsf{ und }}} \sum_{k_i \in K} w_i \geq W
    \end{equation*}

    \vspace*{0.3cm}
    Es kann gezeigt werden, dass das Lösen der einen automatisch auch die andere Variante löst.

\end{document}
