\documentclass{scrartcl}% siehe <http://www.komascript.de>
% packages ---------------------------------------------------------------------------------------------------- packages
\usepackage[margin=3cm]{geometry}
\usepackage[utf8]{inputenc}
\usepackage[T1]{fontenc}
\usepackage[ngerman]{babel}
\usepackage{lmodern}
\usepackage{amsmath}
\usepackage{amssymb}
\usepackage{amsfonts}
\usepackage{cancel}
\usepackage{listings}
\usepackage{multirow}
\usepackage{hhline}
\usepackage{tikz}
\usepackage{pgfplots}
\usepackage[all,dvips,arc,curve,color,frame]{xy}
\usepackage[yyyymmdd,hhmm]{datetime}
\usepackage{float}
\usepackage{hyperref}
\usepackage[table]{xcolor,colortbl}
\usepackage{graphicx}
\usepackage{subcaption}
\usepackage{tabularx}
\usepackage{footnote}
\usepackage{multicol}
\usepackage{stackengine}
\usepackage[norndcorners,customcolors,shade]{hf-tikz}
\usepackage{siunitx}
\usepackage{algorithm}
\usepackage[]{algorithmic}
\usepackage{centernot}
\usepackage{amsthm}
% \usepackage[noend]{algorithmic}

% pgf plotsets -------------------------------------------------------------------------------------------- pgf plotsets
\pgfplotsset{ticks=none}
\pgfplotsset{width=10cm, compat=1.5.1}

% tikz libraries ---------------------------------------------------------------------------------------- tikz libraries
\usetikzlibrary{shapes}
\usetikzlibrary{positioning}
\usetikzlibrary{shapes.misc}
\usetikzlibrary{shapes.arrows,calc,quotes,babel}
\usetikzlibrary{positioning,arrows,chains,scopes,fit}
\usetikzlibrary{matrix,backgrounds}
\usetikzlibrary{trees}
\usetikzlibrary{calc}
\usetikzlibrary{tikzmark}
\usetikzlibrary{intersections}

\sisetup{
locale = DE ,
per-mode = symbol
}

\algsetup{indent=2em}
\newlength\myindent
\setlength\myindent{2em}
\newcommand\bindent{%
    \begingroup
    \setlength{\itemindent}{\myindent}
    \addtolength{\algorithmicindent}{\myindent}
}
\newcommand\eindent{\endgroup}

% commands ---------------------------------------------------------------------------------------------------- commands
\setlength{\parindent}{0pt}
\newcommand{\includepic}[2]{\includegraphics[width=#2\textwidth,height=#2\textheight,keepaspectratio]{#1}}
\newcommand\circlearound[1]{%
\tikz[baseline]\node[draw,shape=circle,anchor=base] {#1} ;}
\newcommand{\emptyrow}[0]{\multicolumn{1}{c}{} & \multicolumn{1}{c}{} & \multicolumn{1}{c}{} & \multicolumn{1}{c}{} &
\multicolumn{1}{c}{} & \multicolumn{1}{c}{} & \multicolumn{1}{c}{} & \multicolumn{1}{c}{} &
\multicolumn{1}{c}{} & \multicolumn{1}{c|}{} & & \\}
\newcommand\underbracetext[1]{\raisebox{1ex}{\ensuremath{\underbrace{\hphantom{\text{#1}}}_{\text{\normalsize #1}}}}}
\newcommand{\headerline}[3]{
\subject{#2}
\title{#1}
\subtitle{#3}
\date{Letztes Update: \today \ - \currenttime \ Uhr}
\maketitle
}
\newcommand{\header}[3]{\section*{#1: #3 - #2}\subsection*{Letztes Update: \today \ - \currenttime \ Uhr}\vspace*{1cm}}
\newcommand{\newproof}[2]{
\vspace*{0.3cm} \textbf{\textsf{Satz:}} #1

\vspace*{0.3cm} \textbf{\textsf{Beweis:}} #2 \vspace*{0.3cm}
}
%\newcommand{\uparrow}[0]{\rlap{\scalebox{1.6}{$\uparrow$}}}
\newlength\dlf
\newcommand{\proofend}[0]{\hfill $\square$}
\newcommand\drawbi[7][4pt]{%
    \path(#2)--(#3)coordinate[at start](h1)coordinate[at end](h2);
    \draw[#4]($(h1)!#1!90:(h2)$)-- node [auto=left] {#5} ($(h2)!#1!-90:(h1)$);
    \draw[#6]($(h1)!#1!-90:(h2)$)-- node [auto=right] {#7} ($(h2)!#1!90:(h1)$);
    }

% colors -------------------------------------------------------------------------------------------------------- colors
\definecolor{amethyst}{rgb}{0.6, 0.4, 0.8}
\definecolor{anti-flashwhite}{rgb}{0.95, 0.95, 0.96}
\definecolor{applegreen}{rgb}{0.55, 0.71, 0.0}
\definecolor{babyblue}{rgb}{0.54, 0.81, 0.94}
\definecolor{bistre}{rgb}{0.24, 0.17, 0.12}
\definecolor{buff}{rgb}{0.94, 0.86, 0.51}
\definecolor{dodgerblue}{rgb}{0.12, 0.56, 1.0}
\definecolor{icterine}{rgb}{0.99, 0.97, 0.37}
\definecolor{tan}{rgb}{0.82, 0.71, 0.55}
\definecolor{upforestgreen}{rgb}{0.0, 0.27, 0.13}

% listing colors
\definecolor{listingBlue}{rgb}{0.13,0.13,1}
\definecolor{listingGreen}{rgb}{0,0.5,0}
\definecolor{listingRed}{rgb}{0.9,0,0}
\definecolor{listingGrey}{rgb}{0.46,0.45,0.48}

% listings ---------------------------------------------------------------------------------------------------- listings
\lstset{basicstyle=\ttfamily}
\lstset{literate=%
{Ö}{{\"O}}1
{Ä}{{\"A}}1
{Ü}{{\"U}}1
{ß}{{\ss}}1
{ü}{{\"u}}1
{ä}{{\"a}}1
{ö}{{\"o}}1
}
\lstset{language=Java,
showspaces=false,
showtabs=false,
breaklines=true,
showstringspaces=false,
breakatwhitespace=true,
commentstyle=\color{listingGreen},
keywordstyle=\color{listingBlue},
stringstyle=\color{listingRed},
basicstyle=\ttfamily,
moredelim=[il][\textcolor{listingGrey}]{$$},
moredelim=[is][\textcolor{listingGrey}]{\%\%}{\%\%}
}
\lstdefinelanguage{Kotlin}{
  keywords={package, as, typealias, this, super, val, var, fun, for, null, true, false, is, in, throw, return, break, continue, object, if, try, else, while, do, when, yield, typeof, yield, typeof, class, interface, enum, object, override, public, private, get, set, import, abstract, procedure},
  keywordstyle=\color{listingBlue}\bfseries,
  ndkeywords={@Deprecated, Iterable, Int, Integer, Float, Double, String, Runnable, dynamic},
  ndkeywordstyle=\color{amethyst}\bfseries,
  emph={println, return@, forEach,},
  emphstyle={\color{orange}},
  identifierstyle=\color{black},
  sensitive=true,
  commentstyle=\color{listingGreen}\ttfamily,
  comment=[l]{//},
  morecomment=[s]{/*}{*/},
  stringstyle=\color{listingRed}\ttfamily,
  morestring=[b]",
  morestring=[s]{"""*}{*"""},
}

% misc ------------------------------------------------------------------------------------------------------------ misc
\graphicspath{ {../lectures-img/} }
\setlength{\parindent}{0pt}

\newcolumntype{C}[1]{>{\centering\arraybackslash}p{#1}} % zentriert mit Breitenangabe
\newcolumntype{L}[1]{>{\raggedright\arraybackslash}p{#1}} % rechtsbündig mit Breitenangabe
\newcolumntype{R}[1]{>{\raggedleft\arraybackslash}p{#1}} % rechtsbündig mit Breitenangabe

% text ------------------------------------------------------------------------------------------------------------ text
\newcommand{\fett}[1]{\textbf{\textsf{#1}}}
\newcommand{\kursiv}[1]{\textit{#1}}

% proofs -------------------------------------------------------------------------------------------------------- proofs
\newtheoremstyle{dotless}{}{}{\itshape}{}{\bfseries\sffamily}{}{ }{}
\theoremstyle{dotless}
\newtheorem*{satz}{Satz:}
\renewenvironment{proof}{{\bfseries \sffamily Beweis:}}{\hfill $\square$}


\usepackage{diagbox}

\definecolor{cadmiumred}{rgb}{0.89, 0.0, 0.13}
\definecolor{cadmiumorange}{rgb}{0.93, 0.53, 0.18}
\definecolor{cadmiumyellow}{rgb}{1.0, 0.96, 0.0}
\definecolor{cadmiumgreen}{rgb}{0.0, 0.42, 0.24}
\definecolor{blizzardblue}{rgb}{0.67, 0.9, 0.93}
\definecolor{persianblue}{rgb}{0.11, 0.22, 0.73}
\definecolor{darkviolet}{rgb}{0.58, 0.0, 0.83}

\begin{document}
    \headerline{Algorithmen und Berechenbarkeit}{Vorlesungsmitschrift}{Vorlesung 15}

    \section*{Satz von Rice}
    TMs halten nicht auf jeder Eingabe, dass bedeutet, sie berechnen partielle Funktionen, die die folgende Form aufweisen:

    \begin{align*}
        & f : \{0,1\}^* \rightarrow \{0,1\}^* \cup \{\colorbox{blue!30}{$\bot$} \} \\
        \rightarrow &\ \colorbox{blue!30}{$\bot$} \text{ bedeutet, TM hält nicht}
    \end{align*}

    Im Rahmen der Vorlesung liegt der Hauptfokus auf TMs, die \texttt{JA}/\texttt{NEIN} ausgeben, also Funktionen der Form:

    \begin{align*}
        & f : \{0,1\}^* \rightarrow \{\colorbox{green!30}{$0$},\colorbox{red!30}{$1$}\}^* \cup \{\bot \} \\
        \rightarrow &\ \colorbox{green!30}{$0$} \text{ TM verwirft Eingabe} \\
        \rightarrow &\ \colorbox{red!30}{$1$} \text{ TM akzeptiert Eingabe}
    \end{align*}

    Sei nun die Menge aller Funktionen, die von irgendeiner TM erkannt werden, definiert als

    \begin{equation*}
        \mathcal{R} = \{f_M : \{0,1\}^* \rightarrow \{0,1\}^* \cup \{\bot\}\ |\text{ M ist TM} \}
    \end{equation*}

    dann besagt der
    \textbf{\textsf{Satz von Rice:}}

    \vspace*{0.3cm}
    Sei $\mathcal{S}$ eine Teilmenge von $\mathcal{R}$ mit $ \emptyset \neq \mathcal{S} \neq \mathcal{R}$, dann ist die Sprache

    \begin{equation*}
        L(\mathcal{S}) = \{ \text{<M> } | \text{ M berechnet eine Funktion aus } \mathcal{S} \}
    \end{equation*}

    unentscheidbar.\proofend

    \vspace*{0.3cm}
    \textbf{\textsf{Beweis:}} Man nimmt an, eine TM $\mathcal{M}_{\mathcal{R}(\mathcal{S})}$ existiert,
    die $L(\mathcal{S})$ entscheidet.
    Dann konstruiert man daraus eine TM $M_{\epsilon}$, die $\mathcal{H}_\epsilon$ entscheidet,
    was im Widerspruch zur Unentscheidbarkeit von $\mathcal{H}_\epsilon$ steht.

    \vspace*{0.3cm}
    Sei $u$ die überall undefinierte Funktion. Man unterscheidet nun die Fälle
    \begin{itemize}
        \item [a)] $u \notin S$
        \item [b)] $u \in S$
    \end{itemize}

    \newpage
    Sein nun $f \neq u$ eine Funktion aus $\mathcal{S}$. Die TM $\mathcal{M}_\epsilon$ arbeitet wie folgt:
    \begin{enumerate}
        \item Falls die Eingabe <M> keine korrekte Kodierung einer TM ist, wird verworfen.
        \item Es wird eine TM $\mathcal{M}^*$ mit folgendem Verhalten konstruiert:
        \begin{itemize}
            \item Man ignoriert die Eingabe $x$, d.h. man simuliert $\mathcal{M}$ auf der Eingabe $\epsilon$.
            \item Man berechnet $f(x)$, d.h. man simuliert $\mathcal{M}_f$ auf der Eingabe $x$.
            \item [$\Rightarrow$] Beobachtung: $\mathcal{M}^*$ berechnet genau dann $f(x)$,
            falls $\mathcal{M}$ auf $\epsilon$ hält. Sonst gilt $f_{\mathcal{M}} = u$.
        \end{itemize}
        \item Man startet die TM $\mathcal{M}_{L(\mathcal{S})}$ auf der Eingabe <$\text{M}^*$> und akzeptiert genau dann,
        wenn $\mathcal{M}_{L(\mathcal{S})}$ akzeptiert.
    \end{enumerate}

    Korrektheit der Konstruktion:
    \begin{itemize}
        \item $w \in \mathcal{H}_\epsilon$
        \begin{itemize}
            \item [$\Rightarrow$] $\mathcal{M}$ hält auf $\epsilon$
            \item [$\Rightarrow$] <$\text{M}^*$> $\in L(\mathcal{S})$
            \item [$\Rightarrow$] $\mathcal{M}_{L(\mathcal{S})}$ akzpetiert <$\text{M}^*$>
            \item [$\Rightarrow$] $\mathcal{M}_{\epsilon}$ akzpetiert $w$
        \end{itemize}
        \item $w \notin \mathcal{H}_\epsilon$
        \begin{itemize}
            \item [$\Rightarrow$] $\mathcal{M}$ hält nicht auf $\epsilon$
            \item [$\Rightarrow$] $\mathcal{M}^*$ berechnet $u$
            \item [$\Rightarrow$] <$\text{M}^*$> $\notin L(\mathcal{S})$
            \item [$\Rightarrow$] $\mathcal{M}_{L(\mathcal{S})}$ verwirft <$\text{M}^*$>
            \item [$\Rightarrow$] $\mathcal{M}_{\epsilon}$ verwirft $w$
        \end{itemize}
    \end{itemize}

    Analog kann der Fall $u \in \mathcal{S}$ gezeigt werden (mit Invertierung des Akzeptanzverhaltens).\proofend

    \vspace*{0.3cm}
    Als Folge des Satzes von Rice gilt insbesondere, dass man nicht entscheiden kann,
    ob ein Programm auf jeder Eingabe hält.

    \begin{equation*}
        S = \{ f: \{0,1\}^* \rightarrow \{0,1\}^* \}
    \end{equation*}

    \section*{Semi-Entscheidbarkeit}

    \textbf{\textsf{Definition Entscheidbarkeit:}}
    Eine Sprache $L$ wird von einer TM $\mathcal{M}$ \textit{\textbf{\textsf{entschieden}}}, falls $\mathcal{M}$ auf jeder Eingabe hält und genau die Wörter aus $L$ akzeptiert.

    \vspace*{0.3cm}
    \textbf{\textsf{Definition Erkennen:}}
    Eine Sprache $L$ wird von einer TM $\mathcal{M}$ \textit{\textbf{\textsf{erkannt}}}, wenn $\mathcal{M}$ jedes Wort aus $L$ akzeptiert und $\mathcal{M}$ kein Wort akzeptiert,
    das nicht in $L$ ist. Auf Eingaben, die nicht in $L$ sind, muss $\mathcal{M}$ nicht halten (Entweder \texttt{NEIN} sagen, oder ewig loopen).

    \vspace*{0.3cm}
    \textbf{\textsf{Definition Semi-Entscheidbarkeit:}}
    Eine Sprache heißt \textit{\textbf{\textsf{semi-entscheidbar}}}, wenn es eine TM $\mathcal{M}$ gibt, die $L$ erkennt.

    \vspace*{0.3cm}
    \textbf{\textsf{Bsp:}} Das Halteporblem ($\mathcal{H} = \{\text{<M>}w\ | \text{ TM }\mathcal{M} \text{ hält auf Eingabe } w\}$) ist semi-entscheidbar, denn es gilt
    \begin{itemize}
        \item Wenn die Eingabe nicht der Form <M>$w$ entspricht, wird verworfen,
        \item Man simuliert $\mathcal{M}$ auf $w$:
        \begin{itemize}
            \item Falls $\mathcal{M}$ hält, wird akzeptiert.
            \item Falls $\mathcal{M}$ nicht hält, wird ewig weiter simuliert.
        \end{itemize}
    \end{itemize}

    \textit{Wenn eine Sprache nicht semi-entscheidbar ist, dann kann sie auch nicht entscheidbar sein. Alle Sprachen, die entscheidbar sind, sind automatisch auch semi-entscheidbar.}

    \section*{Rekursive Aufzählbarkeit}

    \textbf{\textsf{Definition Aufzähler:}}
    Ein Aufzähler für eine Sprache $L$ ist eine TM mit Drucker (Ausgabeband, auf dem der Kopf nur nach rechts bewegt werden darf).
    Ein Aufzähler für $L$ gibt alle Wörter aus $L$ auf dem Drucker aus:
    \begin{itemize}
        \item Es werden nur Wörter aus $L$ ausgedruckt.
        \item Jedes Wort aus $L$ wird irgendwann ausgedruckt.
    \end{itemize}

    \vspace*{0.3cm}
    \textbf{\textsf{Definition Rekursiv aufzählbar:}}
    Eine Sprache $L$ heißt rekursiv aufzählbar, wenn es einen Aufzähler für $L$ gibt.

    \vspace*{0.3cm}
    \textbf{\textsf{Satz:}}
    Eine Sprache $L$ ist genau dann semi-entscheidbar, wenn $L$ rekursiv aufzählbar ist.

    \vspace*{0.3cm}
    \textbf{\textsf{Beweis:}}
    \begin{itemize}
        \item [] \textbf{"`aufzählbar $\Rightarrow$ semi-entscheidbar"':}
        \item [] Man konstruiert eine TM $\mathcal{M}$, die auf einer separaten Spur den Aufzähler simuliert.
        Immer wenn ein Wort ausgegeben wird, wird es mit der Eingabe verglichen.
        Man akzeptiert, falls die Eingabe einem aufgezählten Wort entspricht.
        $\mathcal{M}$ erkennt sonst $L$, da jedes Wort aus $L$ irgendwann aufgezählt wird.
        \item [] \textbf{"`semi-entscheidbar $\Rightarrow$ aufzählbar"':}
        \item [] Seien $w_1, w_2, w_3 \dots$ alle Wörter aus $\Sigma^*$ in kanonischer Reihenfolge (z.B. lexikografisch) sortiert.
        Für $i = 1,2,3, \dots$ simuliert man nun $i$-Schritte von $\mathcal{M}$ auf $w_1, w_2, \dots, w_i$.
        Wird dabei ein Wort akzeptiert, druckt man es aus.
        Offensichtlich werden nur Wörter aus $L$ ausgedruckt.
        Sei $w=w_K \in L$, dann wird $w_k$ von $\mathcal{M}$ nach $t_k$ Schritten akzeptiert.
        Wenn $i = max(k, E_k)$, dann wird $w_k$ betrachtet und $\mathcal{M}$ lange genug simuliert, um $w_k$ zu akzeptieren
        $\Rightarrow$ $w_k$ wird gedruckt.\proofend
    \end{itemize}

    \section*{Eigenschaften entscheidbarer / semi-entscheidbarer Sprachen}
    \textbf{\textsf{Satz:}}
    \begin{itemize}
        \item [a)] Wenn die Sprachen $L_1$ und $L_2$ entscheidbar sind, so ist auch $L_1 \cap L_2$ entscheidbar.
        \item [b)] Wenn die Sprachen $L_1$ und $L_2$ semi-entscheidbar sind, so ist auch $L_1 \cap L_2$ semi-entscheidbar.
    \end{itemize}

    \vspace*{0.3cm}
    \textbf{\textsf{Satz:}}
    \begin{itemize}
        \item [a)] Wenn die Sprachen $L_1$ und $L_2$ entscheidbar sind, so ist auch $L_1 \cup L_2$ entscheidbar.
        \item [b)] Wenn die Sprachen $L_1$ und $L_2$ semi-entscheidbar sind, so ist auch $L_1 \cup L_2$ semi-entscheidbar.
    \end{itemize}

    \vspace*{0.3cm}
    \textbf{\textsf{Satz:}} Wenn $L$ entscheidbar ist, so ist auch $\overline{L}$ entscheidbar.

    \vspace*{0.3cm}
    \textbf{\textsf{Beweis:}} Man invertiert die Ausgabe des Entscheiders für $L$.

    \vspace*{0.7cm}
    \textbf{\textsf{Satz:}} Sind $L$ und $\overline{L}$ semi-entscheidbar, so ist $L$ entscheidbar.

    \vspace*{0.3cm}
    \textbf{\textsf{Beweis:}} Man simuliert Semi-Entscheider für $L$ und $\overline{L}$ parallel.
    Die Entscheidung ist klar, sobald einer der Entscheider akzeptiert. Einer davon muss akzeptieren, denn es gilt:
    $w \in L$ oder $ w \notin L$.

    \vspace*{0.7cm}
    \textbf{\textsf{Korollar:}} $L$ ist unentscheidbar genau dann, wenn mindestens einer der beiden Sprachen $L$
    und $\overline{L}$ nicht semi-entscheidbar ist.

    \vspace*{0.3cm}
    \textbf{\textsf{Bsp:}} Sei $\mathcal{H}$ semi-entscheidbar. Dann gilt: $\overline{\mathcal{M}}$ ist nicht semi-entscheidbar oder aufzählbar.

    \section*{Technik der Reduktion}
    Das Ziel ist es, mithilfe von bereits untersuchten Sprachen zu zeigen,
    ob eine Sprache $L$ entscheidbar oder semi-entscheidbar ist.
    Die Reduktion bildet also die Eingaben eines Problems auf Eingaben eines anderen Problems ab.

    \subsection*{Eingabe-Eingabe-Reduktion}
    \textbf{\textsf{Definition:}} Es seien $L_1$ und $L_2$ Sprachen über $\Sigma^*$.
    Dann heißt $L_1$ auf $L_2$ reduzierbar ($L_1 \leq L_2$), wenn es eine berechenbare Funktion

    \begin{equation*}
        f: \Sigma^* \rightarrow \Sigma^*
    \end{equation*}

    gibt, sodass für alle $x \in \Sigma^*$ gilt:

    \begin{equation*}
        x \in L_1 \qquad \Leftrightarrow \qquad f(x) \in L_2
    \end{equation*}

    Aus "`Berechenbarkeitssicht"' bedeutet das inbesondere, dass $L_1$ nicht schwieriger zu untersuchen ist als $L_2$.
    Beispielhaft also: Falls $L_1$ unentscheidbar und $L_1 \leq L_2$ gilt, so ist auch $L_2$ unentscheidbar.

    \vspace*{0.3cm}
    \textbf{\textsf{Lemma:}} Falls $L_1 \leq L_2$ und $L_2$ entscheidbar, so ist auch $L_1$ entscheidbar.

    \vspace*{0.3cm}
    \textbf{\textsf{Beweis:}} Man konstruiert eine TM $\mathcal{M}_1$,
    die $L_1$ entscheidet unter Nutzung der TM $\mathcal{M}_2$, die $L_2$ entscheidet.
    Nun berechnet man mit der Eingabe $x$ die Funktion $f(x)$ und simuliert $\mathcal{M}_2$ auf $f(x)$ wobei man deren Akzeptanzverhalten übernimmt.

    Dann gilt
    \begin{align*}
        \mathcal{M}_1 \text{ akzpetiert } & \quad \Leftrightarrow \quad \mathcal{M}_2 \text{ akzpetiert } f(x) \\
        & \quad \Leftrightarrow \quad f(x) \in L_2 \\
        & \quad \Leftrightarrow \quad x \in L_1 \\
    \end{align*}\proofend

    \vspace*{0.3cm}
    Aus dem Lemma folgt außerdem: Falls $L_1 \leq L_2$ und $L_1$ unentscheidbar $\quad \Rightarrow L_2$ unentscheidbar.

\end{document}
