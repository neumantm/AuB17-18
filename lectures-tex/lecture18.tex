\documentclass{scrartcl}% siehe <http://www.komascript.de>
% packages ---------------------------------------------------------------------------------------------------- packages
\usepackage[margin=3cm]{geometry}
\usepackage[utf8]{inputenc}
\usepackage[T1]{fontenc}
\usepackage[ngerman]{babel}
\usepackage{lmodern}
\usepackage{amsmath}
\usepackage{amssymb}
\usepackage{amsfonts}
\usepackage{cancel}
\usepackage{listings}
\usepackage{multirow}
\usepackage{hhline}
\usepackage{tikz}
\usepackage{pgfplots}
\usepackage[all,dvips,arc,curve,color,frame]{xy}
\usepackage[yyyymmdd,hhmm]{datetime}
\usepackage{float}
\usepackage{hyperref}
\usepackage[table]{xcolor,colortbl}
\usepackage{graphicx}
\usepackage{subcaption}
\usepackage{tabularx}
\usepackage{footnote}
\usepackage{multicol}
\usepackage{stackengine}
\usepackage[norndcorners,customcolors,shade]{hf-tikz}
\usepackage{siunitx}
\usepackage{algorithm}
\usepackage[]{algorithmic}
\usepackage{centernot}
\usepackage{amsthm}
% \usepackage[noend]{algorithmic}

% pgf plotsets -------------------------------------------------------------------------------------------- pgf plotsets
\pgfplotsset{ticks=none}
\pgfplotsset{width=10cm, compat=1.5.1}

% tikz libraries ---------------------------------------------------------------------------------------- tikz libraries
\usetikzlibrary{shapes}
\usetikzlibrary{positioning}
\usetikzlibrary{shapes.misc}
\usetikzlibrary{shapes.arrows,calc,quotes,babel}
\usetikzlibrary{positioning,arrows,chains,scopes,fit}
\usetikzlibrary{matrix,backgrounds}
\usetikzlibrary{trees}
\usetikzlibrary{calc}
\usetikzlibrary{tikzmark}
\usetikzlibrary{intersections}

\sisetup{
locale = DE ,
per-mode = symbol
}

\algsetup{indent=2em}
\newlength\myindent
\setlength\myindent{2em}
\newcommand\bindent{%
    \begingroup
    \setlength{\itemindent}{\myindent}
    \addtolength{\algorithmicindent}{\myindent}
}
\newcommand\eindent{\endgroup}

% commands ---------------------------------------------------------------------------------------------------- commands
\setlength{\parindent}{0pt}
\newcommand{\includepic}[2]{\includegraphics[width=#2\textwidth,height=#2\textheight,keepaspectratio]{#1}}
\newcommand\circlearound[1]{%
\tikz[baseline]\node[draw,shape=circle,anchor=base] {#1} ;}
\newcommand{\emptyrow}[0]{\multicolumn{1}{c}{} & \multicolumn{1}{c}{} & \multicolumn{1}{c}{} & \multicolumn{1}{c}{} &
\multicolumn{1}{c}{} & \multicolumn{1}{c}{} & \multicolumn{1}{c}{} & \multicolumn{1}{c}{} &
\multicolumn{1}{c}{} & \multicolumn{1}{c|}{} & & \\}
\newcommand\underbracetext[1]{\raisebox{1ex}{\ensuremath{\underbrace{\hphantom{\text{#1}}}_{\text{\normalsize #1}}}}}
\newcommand{\headerline}[3]{
\subject{#2}
\title{#1}
\subtitle{#3}
\date{Letztes Update: \today \ - \currenttime \ Uhr}
\maketitle
}
\newcommand{\header}[3]{\section*{#1: #3 - #2}\subsection*{Letztes Update: \today \ - \currenttime \ Uhr}\vspace*{1cm}}
\newcommand{\newproof}[2]{
\vspace*{0.3cm} \textbf{\textsf{Satz:}} #1

\vspace*{0.3cm} \textbf{\textsf{Beweis:}} #2 \vspace*{0.3cm}
}
%\newcommand{\uparrow}[0]{\rlap{\scalebox{1.6}{$\uparrow$}}}
\newlength\dlf
\newcommand{\proofend}[0]{\hfill $\square$}
\newcommand\drawbi[7][4pt]{%
    \path(#2)--(#3)coordinate[at start](h1)coordinate[at end](h2);
    \draw[#4]($(h1)!#1!90:(h2)$)-- node [auto=left] {#5} ($(h2)!#1!-90:(h1)$);
    \draw[#6]($(h1)!#1!-90:(h2)$)-- node [auto=right] {#7} ($(h2)!#1!90:(h1)$);
    }

% colors -------------------------------------------------------------------------------------------------------- colors
\definecolor{amethyst}{rgb}{0.6, 0.4, 0.8}
\definecolor{anti-flashwhite}{rgb}{0.95, 0.95, 0.96}
\definecolor{applegreen}{rgb}{0.55, 0.71, 0.0}
\definecolor{babyblue}{rgb}{0.54, 0.81, 0.94}
\definecolor{bistre}{rgb}{0.24, 0.17, 0.12}
\definecolor{buff}{rgb}{0.94, 0.86, 0.51}
\definecolor{dodgerblue}{rgb}{0.12, 0.56, 1.0}
\definecolor{icterine}{rgb}{0.99, 0.97, 0.37}
\definecolor{tan}{rgb}{0.82, 0.71, 0.55}
\definecolor{upforestgreen}{rgb}{0.0, 0.27, 0.13}

% listing colors
\definecolor{listingBlue}{rgb}{0.13,0.13,1}
\definecolor{listingGreen}{rgb}{0,0.5,0}
\definecolor{listingRed}{rgb}{0.9,0,0}
\definecolor{listingGrey}{rgb}{0.46,0.45,0.48}

% listings ---------------------------------------------------------------------------------------------------- listings
\lstset{basicstyle=\ttfamily}
\lstset{literate=%
{Ö}{{\"O}}1
{Ä}{{\"A}}1
{Ü}{{\"U}}1
{ß}{{\ss}}1
{ü}{{\"u}}1
{ä}{{\"a}}1
{ö}{{\"o}}1
}
\lstset{language=Java,
showspaces=false,
showtabs=false,
breaklines=true,
showstringspaces=false,
breakatwhitespace=true,
commentstyle=\color{listingGreen},
keywordstyle=\color{listingBlue},
stringstyle=\color{listingRed},
basicstyle=\ttfamily,
moredelim=[il][\textcolor{listingGrey}]{$$},
moredelim=[is][\textcolor{listingGrey}]{\%\%}{\%\%}
}
\lstdefinelanguage{Kotlin}{
  keywords={package, as, typealias, this, super, val, var, fun, for, null, true, false, is, in, throw, return, break, continue, object, if, try, else, while, do, when, yield, typeof, yield, typeof, class, interface, enum, object, override, public, private, get, set, import, abstract, procedure},
  keywordstyle=\color{listingBlue}\bfseries,
  ndkeywords={@Deprecated, Iterable, Int, Integer, Float, Double, String, Runnable, dynamic},
  ndkeywordstyle=\color{amethyst}\bfseries,
  emph={println, return@, forEach,},
  emphstyle={\color{orange}},
  identifierstyle=\color{black},
  sensitive=true,
  commentstyle=\color{listingGreen}\ttfamily,
  comment=[l]{//},
  morecomment=[s]{/*}{*/},
  stringstyle=\color{listingRed}\ttfamily,
  morestring=[b]",
  morestring=[s]{"""*}{*"""},
}

% misc ------------------------------------------------------------------------------------------------------------ misc
\graphicspath{ {../lectures-img/} }
\setlength{\parindent}{0pt}

\newcolumntype{C}[1]{>{\centering\arraybackslash}p{#1}} % zentriert mit Breitenangabe
\newcolumntype{L}[1]{>{\raggedright\arraybackslash}p{#1}} % rechtsbündig mit Breitenangabe
\newcolumntype{R}[1]{>{\raggedleft\arraybackslash}p{#1}} % rechtsbündig mit Breitenangabe

% text ------------------------------------------------------------------------------------------------------------ text
\newcommand{\fett}[1]{\textbf{\textsf{#1}}}
\newcommand{\kursiv}[1]{\textit{#1}}

% proofs -------------------------------------------------------------------------------------------------------- proofs
\newtheoremstyle{dotless}{}{}{\itshape}{}{\bfseries\sffamily}{}{ }{}
\theoremstyle{dotless}
\newtheorem*{satz}{Satz:}
\renewenvironment{proof}{{\bfseries \sffamily Beweis:}}{\hfill $\square$}


\usepackage{diagbox}

\definecolor{cadmiumred}{rgb}{0.89, 0.0, 0.13}
\definecolor{cadmiumorange}{rgb}{0.93, 0.53, 0.18}
\definecolor{cadmiumyellow}{rgb}{1.0, 0.96, 0.0}
\definecolor{cadmiumgreen}{rgb}{0.0, 0.42, 0.24}
\definecolor{blizzardblue}{rgb}{0.67, 0.9, 0.93}
\definecolor{persianblue}{rgb}{0.11, 0.22, 0.73}
\definecolor{darkviolet}{rgb}{0.58, 0.0, 0.83}

\begin{document}
    \headerline{Algorithmen und Berechenbarkeit}{Vorlesungsmitschrift}{Vorlesung 18}

    \section*{Die Komplexitätsklassen $\mathcal{N}\mathcal{P}$ und $\mathcal{P}$}

    \begin{figure}[H]
        \centering
        \begin{table}[H]
            \centering
            \renewcommand{\arraystretch}{1.7}
            \setlength{\tabcolsep}{4mm}
            \begin{tabular}{R{3cm}|p{4cm}|p{4cm}}
                & $\mathcal{N}\mathcal{P}$ & $\mathcal{P}$ \\\hline
                TM & nichtdeterministisch & deterministisch \\
                Übergang & Relation & Funktion \\
                Akzeptanz & falls es einen gültigen Berechnungspfad zu einem akzeptierenden Endzustand gibt & falls die TM in einem akzeptierenden Endzustand landet (es gibt nur einen Pfad) \\
                Laufzeit für eine Eingabe der Länge $n$& $t_{\mathcal{M}} =\ $ \textit{Längster kürzester akzeptierender Berechnungspfad} & $t_{\mathcal{M}} =\ $ \textit{Längster Berechnungspfad} \\
                Charakterisierung
                & $\mathcal{N}\mathcal{P} \rightarrow$ Entscheidungsprobleme, für die die NTM $\mathcal{M}_{NTM}$ akzeptiert mit Laufzeit $t_{\mathcal{M}} =\mathcal{O}(n^\alpha)$ \textbf{\textsf{für ein konstantes}} $\alpha$
                & $\mathcal{P} \rightarrow$ Entscheidungsprobleme, für die die DTM $\mathcal{M}_{DTM}$ akzeptiert mit Laufzeit $t_{\mathcal{M}} =\mathcal{O}(n^\alpha)$ \\
                Beispiele & CLIQUE, Knapsack & Sortieren, Graphzusammenhang \\
            \end{tabular}
        \end{table}
    \end{figure}

    \begin{equation*}
        \mathcal{N}\mathcal{P} \supseteq \mathcal{P} ?
    \end{equation*}

    Die Klasse $\mathcal{N}\mathcal{P}$ kann auch wie folgt charakterisiert werden:

    \vspace*{0.3cm}
    \textbf{\textsf{Satz:}}
    Eine Sprache $\mathcal{L}$ ist genau dann in $\mathcal{N}\mathcal{P}$, wenn es einen Polynomzeitalgorithmus $V$ und ein Polynom $p$ gibt mit
    \begin{equation*}
        x \in L \Leftrightarrow \exists y \in \{0,1\}^{*} \ |\ |y| \leq p(|x|)
    \end{equation*}

    und $V$ akzeptiert $y\#x$.

    \newpage
    \textbf{\textsf{Beweisidee:}}
    Falls $\mathcal{L}$ in $\mathcal{N}\mathcal{P}$ ist, existiert eine NTM $\mathcal{M}$ mit $\mathcal{L}(\mathcal{M}) = \mathcal{L}$ und polynomieller Laufzeit.

    Nun modifiziert man $\mathcal{M}$ zu einer deterministischen TM $V$, die bei jeder (config, gekennzeichnet) - Situation,
    in der mehrere Schritte möglich sind, den Leitstring $y$ liest und entsprechend deterministisch handelt.

    Der Leitstring $y$ muss nur polynomiell lang sein, da die NTM $\mathcal{M}$ polynomielle Laufzeit hat \textit{(weitere Informationen im Skript)}.

    \subsubsection*{Bin-Packing (Optimierung: $\mathcal{N}\mathcal{P}$-schwer)}
    \begin{itemize}
        \item [] \textbf{\textsf{Gegeben:}} $b \in \mathbb{N},\quad w_1,w_2, \dots, w_\mathbb{N} \in \{1, \dots, b\}$
        \item [] {\textbf{\textsf{Gegesucht:}} Eine Funktion $f : \{1, \dots, N\} \rightarrow \{1, \dots, k\}$,
        sodass für alle $i \in \{1, \dots, k\}$ gilt
        \begin{equation*}
            \sum_{j \in f^{-1}(i) } w_j \leq b
        \end{equation*}

        und $k$ minimal.
        }
    \end{itemize}

    \subsubsection*{Bin-Packing (Entscheidungsvariante: $\mathcal{N}\mathcal{P}$-vollständig)}
    \begin{itemize}
        \item [] \textbf{\textsf{Gegeben:}} $b \in \mathbb{N},\quad w_1,w_2, \dots, w_\mathbb{N} \in \{1, \dots, b\}$
        \item [] {\textbf{\textsf{Frage:}}} Existiert eine solche Funktion $f$?
    \end{itemize}

    \section*{Travelling Salesperson Problem (TSP)}
    Gegeben sei ein vollständiger gewichteter Graph mit $N$-Knoten. Es soll nun eine Permutation $\pi$ der Knoten gefunden werden, sodass die folgende Gleichung minimal wird:
    \begin{equation*}
        \sum_{i=0}^{n-1} c(\pi(i)), \pi((i+1)\Mod n)
    \end{equation*}

    \textsf{\textbf{"`Finde die billigste Rundtour"'}}

    \vspace*{0.3cm}
    \textbf{\textsf{Satz:}} Die Entscheidungsvarianten von KP, BPP und TSP sind in $\mathcal{N}\mathcal{P}$.

    \vspace*{0.3cm}
    \textbf{\textsf{Beweisidee:}} Man rät eine nichtdeterministische Lösung und verifiziert dann, dass die Lösung in der Tat gültig ist (alles in polynomieller Zeit).

    \vspace*{0.3cm}
    \textbf{\textsf{Satz:}} Wenn die Entscheidungsvariante von KP in polynomieller Zeit lösbar ist, dann auch die Optimierungsvariante.

    \vspace*{0.3cm}
    \textbf{\textsf{Beweisidee:}} Man nimmt an, ein deterministischer-polyzeit-Algorithmus $\mathcal{A}_{\epsilon}$ für die Entscheidungsvariante existiert.
    \begin{itemize}
        \item []{1. \textbf{\textsf{Schritt}}: Man bestimmt den maximal möglichen Rucksackwert durch Binärsuche mittels $\mathcal{A}_\epsilon$ auf die folgende Weise:
        Man fragt $\mathcal{A}_\epsilon$, ob es einen Rucksack mit Wert $\sum_{i=1}^{N} P_i$ gibt, falls nein, halbiert man, $\dots$ Die Anzahl der Iterationen ist
        \begin{equation*}
            \leq \log \sum p_i \leq n
        \end{equation*}

        bei Eingabelängen $\leq 2^n$.
        }
        \newpage
        \item []{2. \textbf{\textsf{Schritt}}: Man betrachtet den Gegenstand $N$ und entscheidet, ob dieser eingespart werden soll, wie folgt:
        \begin{itemize}
            \item [$a)$] Man bestimmt den Max-Wert $a$ für die Gegenstände $1, \dots, N$
            \item [$b)$] Man bestimmt den Max-Wert $b$ für die Gegenstände $1, \dots, N-1$
            \item [$c)$] Falls $a=b$, dann wirft man den Gegenstand $N$ weg, sonst packt man den Gegenstand $N$ ein.
            \item [$\Rightarrow$] Man wiederholt die Schritte $a)-c)$ für die Gegenstände $1, \dots, N-1$
        \end{itemize}
        }
    \end{itemize}

    Es erfordert $\mathcal{O}(n)$ Aufrufe für das Bestimmen des Max-Werts. Insgesamt ist die Laufzeit polynomiell.\proofend

    \vspace*{0.3cm}
    \textbf{\textsf{Satz:}} Für jedes Entscheidungsproblem $L \in \mathcal{N}\mathcal{P}$ gibt es einen
    deterministischen Algorithmus bzw. eine deterministische TM $\mathcal{M}$, der $L$ entscheidet und dessen Worst-Case-Laufzeit beschränkt ist durch $2^{q(n)}$ für ein Polynom $q$.

    \vspace*{0.3cm}
    \textbf{\textsf{Beweisidee:}} Man enumeriert alle möglichen Leitstrings für die Verifizierer, das sind exponentiell viele.

\end{document}
