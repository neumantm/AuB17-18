\documentclass{scrartcl}% siehe <http://www.komascript.de>
% packages ---------------------------------------------------------------------------------------------------- packages
\usepackage[margin=3cm]{geometry}
\usepackage[utf8]{inputenc}
\usepackage[T1]{fontenc}
\usepackage[ngerman]{babel}
\usepackage{lmodern}
\usepackage{amsmath}
\usepackage{amssymb}
\usepackage{amsfonts}
\usepackage{cancel}
\usepackage{listings}
\usepackage{multirow}
\usepackage{hhline}
\usepackage{tikz}
\usepackage{pgfplots}
\usepackage[all,dvips,arc,curve,color,frame]{xy}
\usepackage[yyyymmdd,hhmm]{datetime}
\usepackage{float}
\usepackage{hyperref}
\usepackage[table]{xcolor,colortbl}
\usepackage{graphicx}
\usepackage{subcaption}
\usepackage{tabularx}
\usepackage{footnote}
\usepackage{multicol}
\usepackage{stackengine}
\usepackage[norndcorners,customcolors,shade]{hf-tikz}
\usepackage{siunitx}
\usepackage{algorithm}
\usepackage[]{algorithmic}
\usepackage{centernot}
\usepackage{amsthm}
% \usepackage[noend]{algorithmic}

% pgf plotsets -------------------------------------------------------------------------------------------- pgf plotsets
\pgfplotsset{ticks=none}
\pgfplotsset{width=10cm, compat=1.5.1}

% tikz libraries ---------------------------------------------------------------------------------------- tikz libraries
\usetikzlibrary{shapes}
\usetikzlibrary{positioning}
\usetikzlibrary{shapes.misc}
\usetikzlibrary{shapes.arrows,calc,quotes,babel}
\usetikzlibrary{positioning,arrows,chains,scopes,fit}
\usetikzlibrary{matrix,backgrounds}
\usetikzlibrary{trees}
\usetikzlibrary{calc}
\usetikzlibrary{tikzmark}
\usetikzlibrary{intersections}

\sisetup{
locale = DE ,
per-mode = symbol
}

\algsetup{indent=2em}
\newlength\myindent
\setlength\myindent{2em}
\newcommand\bindent{%
    \begingroup
    \setlength{\itemindent}{\myindent}
    \addtolength{\algorithmicindent}{\myindent}
}
\newcommand\eindent{\endgroup}

% commands ---------------------------------------------------------------------------------------------------- commands
\setlength{\parindent}{0pt}
\newcommand{\includepic}[2]{\includegraphics[width=#2\textwidth,height=#2\textheight,keepaspectratio]{#1}}
\newcommand\circlearound[1]{%
\tikz[baseline]\node[draw,shape=circle,anchor=base] {#1} ;}
\newcommand{\emptyrow}[0]{\multicolumn{1}{c}{} & \multicolumn{1}{c}{} & \multicolumn{1}{c}{} & \multicolumn{1}{c}{} &
\multicolumn{1}{c}{} & \multicolumn{1}{c}{} & \multicolumn{1}{c}{} & \multicolumn{1}{c}{} &
\multicolumn{1}{c}{} & \multicolumn{1}{c|}{} & & \\}
\newcommand\underbracetext[1]{\raisebox{1ex}{\ensuremath{\underbrace{\hphantom{\text{#1}}}_{\text{\normalsize #1}}}}}
\newcommand{\headerline}[3]{
\subject{#2}
\title{#1}
\subtitle{#3}
\date{Letztes Update: \today \ - \currenttime \ Uhr}
\maketitle
}
\newcommand{\header}[3]{\section*{#1: #3 - #2}\subsection*{Letztes Update: \today \ - \currenttime \ Uhr}\vspace*{1cm}}
\newcommand{\newproof}[2]{
\vspace*{0.3cm} \textbf{\textsf{Satz:}} #1

\vspace*{0.3cm} \textbf{\textsf{Beweis:}} #2 \vspace*{0.3cm}
}
%\newcommand{\uparrow}[0]{\rlap{\scalebox{1.6}{$\uparrow$}}}
\newlength\dlf
\newcommand{\proofend}[0]{\hfill $\square$}
\newcommand\drawbi[7][4pt]{%
    \path(#2)--(#3)coordinate[at start](h1)coordinate[at end](h2);
    \draw[#4]($(h1)!#1!90:(h2)$)-- node [auto=left] {#5} ($(h2)!#1!-90:(h1)$);
    \draw[#6]($(h1)!#1!-90:(h2)$)-- node [auto=right] {#7} ($(h2)!#1!90:(h1)$);
    }

% colors -------------------------------------------------------------------------------------------------------- colors
\definecolor{amethyst}{rgb}{0.6, 0.4, 0.8}
\definecolor{anti-flashwhite}{rgb}{0.95, 0.95, 0.96}
\definecolor{applegreen}{rgb}{0.55, 0.71, 0.0}
\definecolor{babyblue}{rgb}{0.54, 0.81, 0.94}
\definecolor{bistre}{rgb}{0.24, 0.17, 0.12}
\definecolor{buff}{rgb}{0.94, 0.86, 0.51}
\definecolor{dodgerblue}{rgb}{0.12, 0.56, 1.0}
\definecolor{icterine}{rgb}{0.99, 0.97, 0.37}
\definecolor{tan}{rgb}{0.82, 0.71, 0.55}
\definecolor{upforestgreen}{rgb}{0.0, 0.27, 0.13}

% listing colors
\definecolor{listingBlue}{rgb}{0.13,0.13,1}
\definecolor{listingGreen}{rgb}{0,0.5,0}
\definecolor{listingRed}{rgb}{0.9,0,0}
\definecolor{listingGrey}{rgb}{0.46,0.45,0.48}

% listings ---------------------------------------------------------------------------------------------------- listings
\lstset{basicstyle=\ttfamily}
\lstset{literate=%
{Ö}{{\"O}}1
{Ä}{{\"A}}1
{Ü}{{\"U}}1
{ß}{{\ss}}1
{ü}{{\"u}}1
{ä}{{\"a}}1
{ö}{{\"o}}1
}
\lstset{language=Java,
showspaces=false,
showtabs=false,
breaklines=true,
showstringspaces=false,
breakatwhitespace=true,
commentstyle=\color{listingGreen},
keywordstyle=\color{listingBlue},
stringstyle=\color{listingRed},
basicstyle=\ttfamily,
moredelim=[il][\textcolor{listingGrey}]{$$},
moredelim=[is][\textcolor{listingGrey}]{\%\%}{\%\%}
}
\lstdefinelanguage{Kotlin}{
  keywords={package, as, typealias, this, super, val, var, fun, for, null, true, false, is, in, throw, return, break, continue, object, if, try, else, while, do, when, yield, typeof, yield, typeof, class, interface, enum, object, override, public, private, get, set, import, abstract, procedure},
  keywordstyle=\color{listingBlue}\bfseries,
  ndkeywords={@Deprecated, Iterable, Int, Integer, Float, Double, String, Runnable, dynamic},
  ndkeywordstyle=\color{amethyst}\bfseries,
  emph={println, return@, forEach,},
  emphstyle={\color{orange}},
  identifierstyle=\color{black},
  sensitive=true,
  commentstyle=\color{listingGreen}\ttfamily,
  comment=[l]{//},
  morecomment=[s]{/*}{*/},
  stringstyle=\color{listingRed}\ttfamily,
  morestring=[b]",
  morestring=[s]{"""*}{*"""},
}

% misc ------------------------------------------------------------------------------------------------------------ misc
\graphicspath{ {../lectures-img/} }
\setlength{\parindent}{0pt}

\newcolumntype{C}[1]{>{\centering\arraybackslash}p{#1}} % zentriert mit Breitenangabe
\newcolumntype{L}[1]{>{\raggedright\arraybackslash}p{#1}} % rechtsbündig mit Breitenangabe
\newcolumntype{R}[1]{>{\raggedleft\arraybackslash}p{#1}} % rechtsbündig mit Breitenangabe

% text ------------------------------------------------------------------------------------------------------------ text
\newcommand{\fett}[1]{\textbf{\textsf{#1}}}
\newcommand{\kursiv}[1]{\textit{#1}}

% proofs -------------------------------------------------------------------------------------------------------- proofs
\newtheoremstyle{dotless}{}{}{\itshape}{}{\bfseries\sffamily}{}{ }{}
\theoremstyle{dotless}
\newtheorem*{satz}{Satz:}
\renewenvironment{proof}{{\bfseries \sffamily Beweis:}}{\hfill $\square$}


\usepackage{diagbox}

\definecolor{cadmiumred}{rgb}{0.89, 0.0, 0.13}
\definecolor{cadmiumorange}{rgb}{0.93, 0.53, 0.18}
\definecolor{cadmiumyellow}{rgb}{1.0, 0.96, 0.0}
\definecolor{cadmiumgreen}{rgb}{0.0, 0.42, 0.24}
\definecolor{blizzardblue}{rgb}{0.67, 0.9, 0.93}
\definecolor{persianblue}{rgb}{0.11, 0.22, 0.73}
\definecolor{darkviolet}{rgb}{0.58, 0.0, 0.83}


\begin{document}
    \headerline{Algorithmen und Berechenbarkeit}{Vorlesungsmitschrift}{Vorlesung 19}

    \section*{$\mathcal{P}$ vs. $\mathcal{N}\mathcal{P}$}

    \begin{figure}[H]
        \centering
        \begin{tikzpicture}[
        thick]
            \draw [fill=persianblue, name path=c1] (0,0) circle (3cm);
            \draw [fill=cadmiumorange,name path=c2] (0,0.5) circle (2cm);
            \draw [fill=cadmiumyellow,name path=c2] (0,1) circle (1cm);

            \draw (0.2,1) ++(120:2cm) -- ++(160:3.5cm) node [fill=white,inner sep=5pt](a){\texttt{EXPTIME}};
            \draw (0.2,-0.7) ++(120:2cm) -- ++(160:3.5cm) node [fill=white,inner sep=5pt](a){$\mathcal{P}$};
            \draw (0.2,-2.4) ++(120:2cm) -- ++(160:3.5cm) node [fill=white,inner sep=5pt](a){$\mathcal{N}\mathcal{P}$};
        \end{tikzpicture}
    \end{figure}

    Eine der bekanntesten, wichtigsten und offenen Fragen der theoretischen Informatik beschäftigt sich ebenfalls mit $\mathcal{P}$ und $\mathcal{N}\mathcal{P}$:
    \begin{equation*}
        \mathcal{N}\mathcal{P} = \mathcal{P}\ ?
    \end{equation*}

    Wie der Grafik entnommen werden kann, gilt offensichtlich $\qquad \mathcal{P} \subseteq \mathcal{N}\mathcal{P}$ \newline und außerdem $\qquad \mathcal{N}\mathcal{P} \subseteq\ $\texttt{EXPTIME}.

    \vspace*{0.3cm}
    \textbf{\textsf{Idee:}} Man zeigt äquivalente Schwere einiger Probleme in $\mathcal{N}\mathcal{P} \quad \rightarrow$ "`$\mathcal{N}\mathcal{P}$-vollständige Probleme"'.
    Dies hätte zur Folge, dass wenn man von einem dieser Probleme zeigen könnte
    \begin{itemize}
        \item [$\in$]$\mathcal{P}$, dann gilt $\Rightarrow \mathcal{N}\mathcal{P} = \mathcal{P}$
        \item [$\notin$]$\mathcal{P}$, dann gilt $\Rightarrow \mathcal{N}\mathcal{P} \neq \mathcal{P}$
    \end{itemize}

    Zunächst wird eine weitere Form der Reduktion eingeführt.

    \vspace*{0.3cm}
    \textbf{\textsf{Definition polynomieller Reduktion:}} $L_1$ und $L_2$ seien zwei Sprachen über $\Sigma_1$ bzw. $\Sigma_2$.
    \newline $L_1$ ist polynomiell reduzierbar auf $L_2$, wenn es ein
    \begin{equation*}
        f : \Sigma_{1}^* \rightarrow \Sigma_{2}^*
    \end{equation*}

    gibt, das in polynomieller Zeit berechnet werden kann mit $x \in L_1 \Leftrightarrow f(x) \in L_2$. \newline Geschrieben
    $L_1 \leq_p L_2$.

    \textbf{\textsf{Lemma:}} $L_1 \leq_p L_2$, $L_2 \in \mathcal{P} \Rightarrow L_1 \in \mathcal{P}$

    \vspace*{0.3cm}
    \textbf{\textsf{Beweis:}} Klar.

    \subsection*{Beispiel polynomieller Reduktion: Coloring $\leq_p$ SAT}

    \vspace*{0.3cm}
    \textbf{\textsf{Definition: Coloring}}
    \begin{itemize}
        \item [] \textbf{\textsf{Gegeben}} sei $G(V,E)$  (ungerichtet) und $k \in \{1, \dots, |V|\}$.
        \item [] {\textbf{\textsf{Frage:}} Gibt es eine Färbung
        \begin{equation*}
            c : V \rightarrow \{1, \dots, K\}
        \end{equation*}

        der Knoten in $G$ mit $k$-Farben, sodass benachbarte Knoten nie dieselben Farben haben?
        }
    \end{itemize}

    \vspace*{0.3cm}
    \textbf{\textsf{Definition: SAT (Satisfiability)}}
    \begin{itemize}
        \item [] \textbf{\textsf{Gegeben}} sei eine aussagenlogische Formel $\phi$ in KNF.
        \item [] {\textbf{\textsf{Frage:}} Ist $\phi$ erfüllbar?
        }
    \end{itemize}

    \vspace*{0.3cm}
    \textbf{\textsf{Satz:}} Coloring $\leq_p$ SAT.

    \vspace*{0.3cm}
    \textbf{\textsf{Beweis:}} Man beschreibt eine Reduktionsfunktion $f(G,K) = \phi$, sodass gilt:
    \begin{equation*}
        G \text{ hat } k\text{-Färbung} \Leftrightarrow \phi \text{ erfüllbar}
    \end{equation*}

    Für jeden Knoten $v \in V$ und jede Farbe $i \in \{1, \dots, K\}$ führt man eine Variable $x_v^i$ ein \newline
    ($x_v^i =$\texttt{true} $\Rightarrow v$ bekommt die Farbe $i$).

    \begin{align*}
        \phi = & \underbrace{\bigwedge_{v \in V}(x_v^1 \lor x_v^2 \lor \dots \lor x_v^k)}_{\text{Knotenbedingung}} \quad
         \underbrace{\bigwedge_{\{u,v\} \in E}\quad \bigwedge_{i \in \{1, \dots, k\}} (\overline{x_u^i} \lor \overline{x_v^i})}_{\text{Kantenbedingung}}
    \end{align*}

    $\phi$ hat die Größe $\mathcal{O}(k \cdot n + k \cdot n) \Rightarrow \mathcal{O}(n^3)$.

    \vspace*{0.3cm}
    Nun muss noch gezeigt werden, dass die Reduktionsfunktion gültig ist.
    \begin{itemize}
        \item [$\Rightarrow$] Sei $c$ eine $k$-Farbe für G: Nun setzt man $x_v^i =$\texttt{true} für $v$ mit $c(v)=i$, sonst $x_v^{i}=$\texttt{false}.
        Die Knotenbedingung ist offensichtlich erfüllt. Auch die Kantenbedingung ist erfüllt, da immer $\overline{x}_u^i \lor \overline{x}_v^i$ gilt,
        weil sonst $v$ und $u$ dieselbe Farbe haben müssten.
        \item [$\Leftarrow$] Angenommen, man hat eine erfüllende Belegung für $\phi$. Für jeden Knoten gibt es also mindestens ein $x_v^{i}=$\texttt{true}.
        Nun wählt man für jeden dadurch eine Farbe aus. Sei $\{u,v\} \in E:$ Angenommen, $c(u)=c(v)=i$.
        Dann wäre jedoch $x_v^i = x_u^i =$\texttt{true} und $\overline{x}_v^i \lor \overline{x}_u^{i}=$\texttt{false}. Daraus folgt, $\phi$ ist nicht erfüllt, woraus gilt $c(u) \neq c(v)$.\proofend
    \end{itemize}

    \vspace*{0.3cm}
    \textbf{\textsf{Korollar:}}
    \begin{itemize}
        \item [$a)$] Wenn SAT in Polynomzeit deterministisch lösbar ist, ist Coloring auch in Polynomzeit lösbar.
        \item [$b)$] Wenn SAT nicht in Polynomzeit deterministisch lösbar ist, ist Coloring auch nicht in Polynomzeit lösbar.
    \end{itemize}

    \vspace*{0.3cm}
    \textbf{\textsf{Definition $\mathcal{N}\mathcal{P}$-hart/$\mathcal{N}\mathcal{P}$-schwer:}} Ein Problem $L$ heißt $\mathcal{N}\mathcal{P}$-hart falls gilt:
    \begin{equation*}
       \forall L' \in \mathcal{N}\mathcal{P} \ :\ L' \leq_p L
    \end{equation*}

    \vspace*{0.3cm}
    \textbf{\textsf{Satz:}} $L$ ist $\mathcal{N}\mathcal{P}$-hart und $L \in \mathcal{P}$, dann gilt $\mathcal{P} = \mathcal{N}\mathcal{P}$.

    \vspace*{0.3cm}
    \textbf{\textsf{Definition $\mathcal{N}\mathcal{P}$-vollständig:}}
    Ein Problem heißt $\mathcal{N}\mathcal{P}$ vollständig, falls
    \begin{enumerate}
        \item $L \in \mathcal{N}\mathcal{P} \qquad\qquad$ (meistens einfacher zu zeigen)
        \item $L$ ist $ \mathcal{N}\mathcal{P}-\text{hart } \qquad$ (meistens schwieriger zu zeigen)
    \end{enumerate}

    \textbf{\textsf{NPC}} ist die Klasse der $\mathcal{N}\mathcal{P}$-vollständigen Probleme.

    \vspace*{0.6cm}
    \textbf{\textsf{Satz:}} SAT ist $\mathcal{N}\mathcal{P}$-vollständig.

    \vspace*{0.3cm}
    \textbf{\textsf{Beweisidee:}}
    \begin{itemize}
        \item [$a)$] SAT $\in \mathcal{N}\mathcal{P}$ ist einfach.
        \item [$b)$] { $\forall L' \in \mathcal{N}\mathcal{P} \ :\ L' \leq_p L$ \newline
        $L' \in \mathcal{N}\mathcal{P}$ heißt, es gibt eine NTM $\mathcal{M}$, die $L'$ in polynomieller Zeit erkennt.
        Nun kodiert man das Verhalten von $\mathcal{M}$ in eine aussagenlogische Formel, die genau dann erfüllbar ist, falls $\mathcal{M}$ die Eingabe in polynomieller Zeit akzeptiert \textit{(mehr Informationen im Skript)}.
        }
    \end{itemize}

    \vspace*{0.3cm}
    \textbf{\textsf{Lemma:}} SAT $\leq_p$ 3SAT

    \vspace*{0.3cm}
    \textbf{\textsf{Satz:}} 3SAT ist $\mathcal{N}\mathcal{P}$-vollständig.

\end{document}
